\section{Contribuciones}
En este trabajo se consiguió plantear modificaciones al paisaje de búsqueda que combinados con una metaheurística de búsqueda local iterada obtuvieron resultados mejores que los métodos simples que se tienen actualmente. 

De todos los cambios que se implementaron al paisaje de búsqueda, el que tuvo más impacto en el desempeño de la búsqueda local iterada consistió en cambiar la representación y con ella cambiar la estructura de vecindad y el conjunto de soluciones representables. No se logró plantear una función de fitness que mostrara una mejora sustancial en los resultados aunque definitivamente tiene un impacto positivo considerar algo más que únicamente el makespan de una planificación para hallar mejores soluciones.

Estas modificaciones pueden servir de punto de partida para proponer estrategias más complejas que puedan acercarse mucho más a los resultados del estado del arte pero en una fracción del tiempo requerido actualmente. También hay trabajo por hacer para refinar la representación y la estructura de vecindad aquí presentadas.


