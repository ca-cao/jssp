\section{Contribuciones}
En este trabajo se consiguió plantear modificaciones al paisaje de búsqueda que combinados con una metaheurística de búsqueda local iterada obtuvieron resultados con una mediana de error relativo no mayor al $0.21$ para todas las instancias de prueba en un tiempo de 5 minutos no paralelo comparado con las 24 a 48 horas de cómputo paralelo necesarios para hallar los resultados del estado del arte.

De todos los cambios que se implementaron al paisaje de búsqueda, el que tuvo más impacto en el desempeño de la búsqueda local iterada consistió en cambiar la representación y con ella el conjunto de soluciones representables y la estructura de vecindad. No se logró plantear una función de fitness que mostrara una mejora sustancial en los resultados aunque definitivamente tiene un impacto positivo considerar algo más que únicamente el makespan de una planificación para hallar mejores soluciones.

Estas modificaciones pueden servir de punto de partida para proponer estrategias más complejas que puedan acercarse mucho más a los resultados del estado del arte pero en una fracción del tiempo requerido actualmente. También hay trabajo por hacer para refinar la representación y la estructura de vecindad aquí presentadas.


