\section{Contribuciones}
En este trabajo se consiguió plantear modificaciones al paisaje de búsqueda que, combinados con la metaheurística búsqueda local iterada, obtuvieron resultados 
con una mediana de error relativo no mayor al $0.21$ para todas las instancias de prueba en un tiempo de 5 minutos no paralelo comparado con las 24 a 48 horas 
de cómputo paralelo necesarios para hallar dichas soluciones.
%
Además, estas mejoras fueron muy significativas en comparación con las casos en que no se incorporan dichas modificaciones, ya que en estos casos el error
relativo se incrementó hasta $0.34$ en las instancias más difíciles.

De todos los cambios que se implementaron al paisaje de búsqueda, el que tuvo más impacto en el desempeño de la búsqueda local iterada consistió en cambiar 
la representación con el fin de generar exclusivamente soluciones que cumplen ciertas propiedades que se saben que aparecen en soluciones óptimas.
%
No se logró plantear una función de fitness que mostrara una mejora sustancial en los resultados aunque definitivamente tiene un impacto positivo considerar 
algo más que únicamente el makespan de una planificación para hallar mejores soluciones.

Las modificaciones propuestas en esta tesis pueden servir como punto de partida para proponer estrategias más complejas que puedan acercarse mucho más a los 
resultados del estado del arte pero en una fracción del tiempo requerido actualmente. 
%
Por su puesto, conseguir esto es un reto importante que requiere refinar aún más la representación y la estructura de vecindad aquí presentadas, así como
estudiar la forma en que se debe integrar con muchas de las componentes que han llevado a encontrar las mejores soluciones conocidas hasta la fecha.


