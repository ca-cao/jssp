\section{Propuestas}
\subsection*{Hipótesis}
Es posible conseguir resultados comparables al estado del arte con algoritmos sencillos y rápidos si modificamos el paisaje de búsqueda de manera adecuada.\\
Como se explica mas adelante, el paisaje de búsqueda es la combinación de tres elementos: función de fitness, espacio de búsqueda y estructura de vecindad. Estos tres elementos tienen un efecto muy importante en el éxito que puede llegar a tener una metaheurística
Para poner a prueba la hipótesis se presentan las siguientes propuestas que serán exploradas mediante experimentos computacionales:
\begin{enumerate}
\item Utilizar la búsqueda local iterada (ILS por sus siglas en inglés) por ser un algoritmo simple y rápido.
\item Agregar nuevos movimientos a la vecindad N7.
\item Crear una función de fitness que no solo tome en  cuenta el makespan sino también otras características de la solución.
\item Utilizar una nueva representación junto con un esquema de decodificación para limitar el espacio de búsqueda.
\item Construir una nueva estructura de vecindad a partir de la nueva representación.
\end{enumerate}
