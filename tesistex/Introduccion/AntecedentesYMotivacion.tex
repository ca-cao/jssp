\section{Antecedentes y Motivación}
% un pequeño review
Los problemas de planificación surgen de manera natural de un sistema de producción o procesamiento en los que la tarea que se quiere completar consiste a su vez de varias subtareas que pueden o deben repartirse entre distintas máquinas o unidades de procesamiento. A grandes rasgos lo que se busca es asignar a cada subtarea una máquina que debe procesarla y un tiempo de inicio y fin.  Estos problemas pueden surgir de todo tipo de contextos, desde la producción de algo como una bicicleta hasta la forma en que los sistemas computacionales procesan información o cómo se asignan las clases en una escuela. En muchos de estos contextos no solo se requiere hallar una planificación sino que a demás se quiere encontrar una que sea óptima en algún sentido, habitualmente se quiere la que haga que se complete el trabajo lo más rápido posible aunque puede haber otros criterios.\\
Desde los década de los 50 se comezaron a formular algoritmos de planificación\cite{johnson1954optimal} y el interés en ellos ha crecido en las décadas siguientes hasta la fecha.\\ 
 
En este trabajo se trata un tipo especial de problema de plaficación conocido como el Problema de Planificación de Producción tipo Taller o JSP por sus siglas en inglés. Este problema consiste en hallar una planificación que tome el minimo tiempo posible para realizar un conjunto de trabajos, cada una de ellos consistiendo de una secuencia de operaciones a procesarse en un orden determinado, en un conjunto de máquinas. Este es un problema de optimización combinatoria en la que se requiere encontrar una secuencia de procesamiento para las operaciones en cada una de las máquinas disponibles y es NP duro cuando el número de máquinas es mayor a dos.\\

Anteriormente se han propuesto métodos exactos para resolver este problema \cite{Brucker1994} pero la cantidad de recursos computacionales que requieren conforme el tamaño del problema (número de máquinas y trabajos) aumenta los hace impráticos excepto para instancias pequeñas.
% poner explicacion de instancias


Dado el interés que se tiene en este problema y en que su tamaño puede ser relativamente grande, se comenzaron a proponer otros métodos aproximados para encontrar soluciones buenas en tiempos aceptables. Estós métodos pueden agruparse de la siguiente manera\cite{Zhang2019}:
\begin{itemize}
\item \textbf{Métodos Constructivos} Estos métodos construyen una planificación mediante el uso de una relga simple por lo que son muy rápidos y conceptualmente sencillos. En general pueden clasificarse en tres tipos: los que usan reglas de prioridad, los que usan heurísticas de cuello de botella y los que usan algún método de inserción. El primero de estos consiste en establecer una forma de elegir la operación a planificar de varias disponibles. Esto puede hacerse por ejemplo eligiendo la que tome más tiempo o la que pueda procesarse antes. El segundo de estos consiste en replantear el problema como una serie de subproblemas más sencillos que puedan resolverse iterativamente hasta que se tenga una solución completa. El tercer tipo construye una solución partiendo del ordenamiento de solo un subconjunto de operaciones y progresivamente agregando más a partir de las que ya se tienen.
\item \textbf{Métodos de inteligencia artificial} En estos métodos utlizan redes neuronales para encontrar la planificación. Existen muchos tipos de redes que se han diseñado para atacar este problema aunque en general suelen combinarse con otros métodos para obtener resultados competitivos.
\item \textbf{Métodos de búsqueda local} En estos métodos se establece una forma de crear soluciones nuevas a partir de una solución dada para reemplazarla una nueva y repetir el proceso hasta que se cumpla algún criterio de paro. Para crear nuevas soluciones se hace un cambio pequeño como, por ejemplo, intercambiar el orden de dos operaciones de modo que sea posible evaluar todas las posibles soluciones generadas al aplicar esta modificación. En general estos métodos se distinguen entre sí por la manera en que se escoge la solución con la cual reemplazar la solución inicial y el criterio de paro.
\item \textbf{Metaheurísticas} Estos métodos combinan elementos de los previamente mencionados para plantear estrategias que obtengan aun mejores resultados. Es muy común que se planteen metaheurísticas inspiradas en la naturaleza como los algoritmos genéticos basados en la evolución o algoritmos basados en el comportamiento de seres vivos. Al ser de alto nivel pueden adaptarse a una gran cantidad de problemas y obtener buenos resultados.   
\end{itemize}

Actualmente los algoritmos más exitosos para resolver el JSP son algoritmos meméticos que combinan un método de búsqueda local con una metaheurística poblacional que mantiene varias soluciones. El método de búsqueda local más usado se conoce como búsqueda tabú que mediante el uso de memoria logra escapar de óptimos locales.

Aunque se han obtenido buenos resultados con estos algoritmos se requieren de 24 a 48 horas paralelas de ejecución. La literatura reciente se ha centrado en hacer más eficiente la búsqueda tabú para reducir los tiempos de ejececución, dejando de lado los otros elementos que forman parte de las metaheutísticas. Dentro de estos elementos hay tres qué son comunes a todas las metaheurísticas de búsqueda local. El primero y más básico es cómo representar computacionalmente una solución, actualmente lo más común es que una solución se represente como un grafo dirigido cuyos nodos son las operaciones a procesar y las aristas indican el ordenamiento de éstas. El segundo es la forma de generar una solucion a partir de otra, esto se conoce como una estructura de vecindad y la más exitosa es conocida como N/ y fue propuesta en 2006 \cite{Zhang2007}. El tercer elemento se conoce como función de fitness o aptitud y nos permite comparar soluciones y decidir si una es mejor que otra.\\

Los tres elementos anteriormente mencionados conforman lo que se conoce como paisaje de búsqueda.
