En este capítulo se presenta el problema de interés así como una revisión de los distintos métodos que se han formulado para resolverlo. 
%
Se plantea la hipótesis del trabajo y los objetivos así como las propuestas que servirán para alcanzar dichos objetivos. 
%
Al final del mismo se resumen las contribuciones de este trabajo.

\section{Antecedentes y Motivación}
Los problemas de planificación surgen de manera natural en sistemas de producción o procesamiento en los que la tarea general que se 
quiere completar está compuesta por varios trabajos que a su vez están compuestos por operaciones que pueden o deben repartirse entre distintos recursos, como máquinas o unidades de procesamiento. 
%
A grandes rasgos un problema de planificación consiste en asignar tiempos de procesamiento a las operaciones antes mencionadas. A esta asignación se le conoce como \textbf{planificación}.
%
Estos problemas pueden surgir de todo tipo de contextos, desde la producción de algo como una bicicleta hasta la forma en que los sistemas computacionales 
procesan información o cómo se asignan las clases en una escuela. 
%
En la mayor parte de estos contextos se tienen restricciones, por ejemplo que ciertas operaciones respeten algún orden de precedencia o que sean procesadas por algún recurso en específico. Se dice que una planificación es factible si cumple con las restricciones impuestas.
%
En la mayor parte de estos contextos no solo se requiere hallar una planificación factible sino que además se quiere 
encontrar una que sea óptima en algún sentido.
%
Una de las funciones objetivo más habitualmente usada consiste en minimizar el tiempo requerido para completar la tarea; sin embargo, existen muchas otras.

Desde los década de los 50 se comenzaron a desarrollar algoritmos para hallar planificaciones óptimas para problemas con dos y tres máquinas~\cite{johnson1954optimal}
y, dado su gran campo de aplicaciones, el interés en ellos ha crecido en las décadas siguientes hasta la fecha.
%
Así, han surgido numerosas formulaciones de problemas de planificación, que incluyen diferentes restricciones y funciones objetivo.
 
En este trabajo se trata un tipo específico de problema de planificación conocido como el \textbf{Problema de Planificación de Producción tipo Taller} o \textbf{JSP} por sus 
siglas en inglés (Job Shop Schedulig Problem). 
%
En este problema la tarea global está constituida por un conjunto de $n$ trabajos que se deben procesar en $m$ máquinas. 
%
Cada uno se esos trabajos están constituido por un conjunto de operaciones (habitualmente $m$), que se deben procesar en un orden determinado y cada una de ellas
en una máquina en específico.
%
El propósito es encontrar una planificación (asignación de las operaciones a las máquinas), que respete el orden establecido y que minimice el tiempo en que finaliza
la última operación (makespan).
%
El JSP es un problema de optimización combinatoria que pertenece a la clase de problemas \textbf{NP-hard} cuando el número de máquinas es mayor a dos\cite{garey1976complexity}.

Anteriormente se han propuesto métodos exactos para resolver el JSP~\cite{Brucker1994}. 
%
Sin embargo, la cantidad de recursos computacionales que requieren conforme el tamaño del problema (número de máquinas y trabajos) aumenta los hace inviables excepto 
para instancias pequeñas de aproximadamente 10 máquinas y 10 trabajos.
%
Dado el interés que se tiene en este problema y en que su tamaño puede ser suficientemente grande para no poder utilizar métodos exactos, se han propuesto métodos aproximados 
para encontrar soluciones aceptables en tiempos reducidos. 
%
Estos métodos pueden agruparse de la siguiente manera\cite{jain1998state,Zhang2019}:
\begin{itemize}
\item \textbf{Métodos Constructivos.} Estos métodos construyen una planificación mediante el uso de alguna regla simple que de forma iterativa va tomando decisiones sobre la solución
hasta generar una solución completa. Se caracterizan por ser muy rápidos y conceptualmente sencillos pero habitualmente no alcanzan calidades cercanas a las óptimas.
Se pueden clasificar en tres tipos: los que usan reglas de prioridad, los que usan heurísticas de cuello de botella y los que usan algún método de inserción.

El primero de estos consiste en establecer una forma de elegir la operación a planificar entre varias disponibles. Esto puede hacerse por ejemplo eligiendo la que tome 
más tiempo o la que pueda procesarse antes.

El segundo consiste en replantear el problema como una serie de subproblemas más sencillos que puedan resolverse iterativamente hasta que se tenga una solución completa, 
por ejemplo planificando una sola máquina manteniendo las otras fijas.

El tercer tipo construye una solución partiendo del ordenamiento de solo un subconjunto de operaciones y progresivamente agregando más a partir de las que ya se tienen.

\item \textbf{Métodos de inteligencia artificial.} En estos métodos se utilizan redes neuronales para encontrar la planificación. Existen muchos tipos de redes que se 
han diseñado para atacar este problema aunque en general suelen combinarse con otros métodos para obtener resultados competitivos, ya que por sí sólas no suele generar
soluciones de muy alta calidad. La gran desventaja de estos métodos es que se tienen que construir y entrenar redes a la medida según el tipo de instancia y a menudo 
se necesita mucha adaptación para aplicarse a nuevas variantes, lo cual los vuelve poco prácticos.

\item \textbf{Métodos de búsqueda local.} En estos métodos se establece una forma de crear soluciones nuevas a partir de una solución dada para reemplazarla por una nueva 
que sea mejor y repetir el proceso hasta que se cumpla algún criterio de paro.  Para crear nuevas soluciones se hace un cambio pequeño como, por ejemplo, intercambiar el 
orden de dos operaciones.  En general estos métodos se distinguen entre sí por la manera de generar nuevas soluciones y por el criterio de paro. Tienen el inconveniente 
de quedarse estancados en óptimos locales. 

\item \textbf{Metaheurísticas.} Estos métodos combinan elementos de los previamente mencionados junto con otras ideas para tratar de evitar óptimos locales. 
Se distingue entre metaheurísticas de trayectoria y metaheurísticas poblaciones, y son los métodos que han dado mejores resultados para las instancias de mayor
tamaño. 
%
Es muy común que se planteen metaheurísticas inspiradas en la naturaleza como los algoritmos genéticos basados en la evolución, o algoritmos basados en el 
comportamiento de seres vivos, como el algoritmo de colonia de hormigas o de abejas.
%
Este tipo de optimizadores son estrategias de alto nivel que se pueden adaptar a diferentes problemas, y prácticamente la totalidad de las metaheurísticas más populares
han sido adaptadas para tratar el JSP.
\end{itemize}

Actualmente los algoritmos más exitosos para resolver el JSP son algoritmos meméticos que combinan un método de búsqueda local o una metaheurística de trayectoria,
con una metaheurística poblacional que mantiene varias soluciones. 
%
En particular, la mataheurística de trayectoria búsqueda tabú es una de las más efectivas en el JSP.
%
Aunque se han obtenido buenos resultados con estos algoritmos, su mayor inconveniente el su alto requerimiento de recursos de cómputo.
%
En particular, los mejores resultados conocidos se han generado haciendo uso de ejecuciones de 24 a 48 horas en entornos paralelos, y los resultados a corto
plazo con estos métodos son de baja calidad.
%
La literatura reciente se ha centrado en hacer más eficiente la búsqueda tabú para reducir los tiempos de ejecución, y aunque tener optimizadores muy efectivos
a largo plazo es importante, también lo es disponer de métodos que a corto plazo sean capaces de encontrar soluciones de alta calidad.

Como se explicará a detalle más adelante un aspecto muy importante a la hora de diseñar metaheurísticas de trayectoria viene dado por la definición del
paisaje de búsqueda del problema, el cual se compone de tres elementos.
%
El primero y más básico es cómo representar computacionalmente una solución en el optimizador, y cómo decodificarla a soluciones del problema a tratar.
%
Actualmente lo más común es que una solución se represente con tantas permutaciones de operaciones como máquinas, y que dichas permutaciones establezcan
el orden exacto en el que las operaciones se ejecutan en la máquina.
%
El segundo es la definición de la vecindad, que permite establecer cómo se generan unas soluciones a partir de otras.
%
En este caso la más exitosa es conocida como N7 y fue propuesta en 2006~\cite{Zhang2007}. 
%
Finalmente, el tercer elemento se conoce como función de fitness o aptitud y nos permite comparar soluciones y decidir si una es mejor que otra. 
%
El análisis de la literatura muestra que durante bastantes años no se han realizado innovaciones con respecto al paisaje de búsqueda y que se ha dado
énfasis en otras componentes, como operadores de cruce, modelos subrogados o estructuras de datos que permitan reducir los tiempos de los métodos híbridos
basados en búsqueda tabú.


