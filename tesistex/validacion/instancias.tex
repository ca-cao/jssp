\section{Conjunto de instancias de prueba}
A lo largo de los años se han propuesto múltiples instancias de prueba para medir el desempeño de los algoritmos para la solución del JSP. Estos conjuntos de prueba han aumentado progresivamente en tamaño y dificultad. 

Actualmente el conjunto más popular se debe a E. Demirkol, S. Mehta y R. Uzsoy \cite{demirkol1997computational}. Consiste en 80 instancias que van desde 20 trabajos y 15 máquinas ($20\times 15$) hasta 50 trabajos y 20 máquinas ($50\times 20$) y se conoce como \textbf{DMU01-80}. La segunda mitad de estas instancias \textbf{DMU40-80} son consideradas especialmente difíciles porque las operaciones iniciales de todos los trabajos tienen que ser procesadas en solo una fracción de las máquinas lo cual genera un cuello de botella al inicio de la planificación.

En este trabajo se ponen a prueba las modificaciones planteadas al comparar los resultados obtenidos con los mejores reportados en la literatura hasta la fecha.

Las modificaciones se presentan en el siguiente orden: en primer lugar se presentan los resultados de búsqueda local con la vecindad N7 y con función de fitness igual al makespan, es decir que el único cambio es en la metaheurística utilizada para tener una linea base. En segundo lugar se presentan las modificaciones hechas a la función de fitness. Posteriormente se presentan los resultados de la extensión a la vecindad N7. Por último se presentan los resultados del cambio de representación junto con la nueva estructura de vecindad basada en soluciones activas.



