\chapter*{Resumen}
Existe una gran cantidad de tipos de problemas de planificación, siendo el problema de planificación de producción tipo taller
(JSP por sus siglas en inglés) un tipo específico que ha atraído considerable atención por su dificultad y aplicabilidad. 
%
En general estos problemas consisten en encontrar una planificación que cumpla un conjunto de restricciones y optimice alguna función
objetivo, siendo la minimización del tiempo requerido para completar un conjunto de trabajos (makespan) una de las funciones objetivo más habituales.
%
En esta tesis se trabaja con la formulación estándar del JSP, en la que precisamente se debe minimizar el makespan, y en particular se analiza 
el efecto sobre el rendimiento de distintas estrategias de modificación del paisaje de búsqueda.

En la actualidad existen algoritmos exactos para encontrar la solución óptima del JSP~\cite{Brucker1994}.
%
Sin embargo, el tiempo requerido para resolver instancias grandes es demasiado elevado como para que sean prácticos o inclusive factibles.
%
La complejidad de este problema lo hizo un candidato ideal para utilizar métodos aproximados con el fin de generar soluciones aceptables 
en tiempos razonables. 
%
Así, en la actualidad, los mejores resultados obtenidos para las instancias de prueba más difíciles se han encontrado mediante metaheurísticas, y
en específico la metaheurística de trayectoria búsqueda tabú es una de las más efectivas.
%
Las metaheurísticas son convenientes por su relativa simplicidad algorítmica frente a otros métodos; sin embargo, el tiempo requerido para obtener 
soluciones de alta calidad también ha comenzado a crecer y el diseño de las técnicas se ha complicado enormemente, incluyendo así numerosas
componentes que no son para nada sencillas de programar e integrar entre sí.
%
A modo de ejemplo, algunos trabajos recientes consideran ejecuciones de varios días en ambientes paralelos, y que integran múltiples componentes
como algoritmos evolutivos, búsqueda tabú y modelos subrogados.
%
Estos requerimientos pueden no ser muy prácticos a la hora de resolver problemas de este tipo en empresas, por lo que esta tesis parte de la
idea de que es importante dar un paso atrás y replantear un poco la forma de aplicar estos métodos para evitar complicarlos en exceso.

En el campo de las metaheurísticas hay varios conceptos fundamentales que tienen un gran impacto en su desempeño.
%
En este trabajo se analizan tres de ellos que juntos componen lo que se conoce como paisaje de búsqueda: la representación de las soluciones, 
la definición de vecindad de una solución, y la función de aptitud o fitness. 
%
En el trabajo se proponen y analizan diferentes alternativas para cada una de estas componentes, y se evalúa el impacto sobre el rendimiento utilizando
para ello una de las metaheurísticas de trayectoria más sencillas, la búsqueda local iterada.
%
Esta evaluación experimental considera las instancias dmu, las cuales son uno de los conjuntos de prueba más utilizados actualmente y que aun no ha
sido resuelto por completo.
%
A partir de los análisis realizados se pudo concluir que de las tres componentes en las que se plantearon modificaciones, la que más impacto tiene para el
rendimiento es la representación de soluciones.
%
Entre otras propuestas, se diseñó una nueva representación basada en algunas propiedades que se estudiaron en los 90s, y junto con definiciones
de funciones de fitness y vecindades adecuadas, permitió mejorar de forma significativa la calidad de las soluciones encontradas en ejecuciones a corto plazo. 
%
Dado que esta representación difiere de las usadas en algoritmos mucho más complejos como búsqueda tabú, los resultados discutidos en esta tesis
abren la puerta a nuevas investigaciones que utilicen este tipo de representación más elaborada, junto a metaheurísticas de
trayectoria más complejas.
%
En esta tesis también se proponen nuevas funciones de fitness y definiciones de vecindad, mostrándose que, aunque permiten mejorar aún más las soluciones,
el efecto es mucho más limitado.
%
Entre las contribuciones, cabe destacar que se dispone de un optimizador, en el que en ejecuciones de 5 minutos en entornos secuenciales, 
la mediana del error relativo es menor a $0.21$ en todas las instancias tratadas, valor que es significativamente inferior en comparación de los casos 
en que no se incorporan los cambios propuestos en esta tesis, que alcanzan errores relativos de $0.34$.


\let\cleardoublepage\clearpage



