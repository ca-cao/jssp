\section{Llaves aleatorias}
La representación propuesta se basa en asignar a cada operación una número real entre 0 y 1 el cual sirve para definir un orden entre operaciones en una misma máquina mediante un proceso de decodificación un planteamiento similar puede encontrarse en \cite{norman1996random}.



Para decodificar la solución a partir de las llaves se utiliza el algoritmo de Giffler \& Thompson \cite{Giffler1960} con el cual se generan soluciones activas. El primer paso es marcar como planificable la primera operacion de cada trabajo.Posteriormente se identifica la operación $O_{min}$ con el menor tiempo de finalización si se planifacara ya y la máquina $m^*=m(O_{min})$ en la cuál debe procesarse. Se toma el tiempo de finalización $t^*_f = t_f(O_{min})$ para escoger alguna de las operaciones planificables en $m^*$ cuyo tiempo de inicio sea menor a $t^*_f$, se actualizan las operaciones planificables y este proceso se repite hasta completar la planificación. A continuación se presenta formalmente el algoritmo.


\begin{algorithm}[H]
 \KwData{Instancia del JSP}
 \KwResult{Planificación activa}
 Inicializar el conjunto de operaciones planificables $\Omega$\;
 \While{$\Omega$ no vacío}{
    Determinar la operación con el menor tiempo de finalización $O_{min}\in\Omega$\;
    Determinar el tiempo de finalazición $t^*_f$ y la máquina $m^*$ en que se procesa $O_{min}$\;
    Escoger $O^*\in\Omega$ tal que $t_i(O^*) < t^*_f$ y $m(O^*)=m^*$\;
    Asignar tiempo de inicio y fin a $O^*$\;
    Actualizar $\Omega$ eliminando a $O^*$ y agregando a su sucesora si existe\;
 }
    \label{alg:GT}
    \caption{Algoritmo de Giffler \& Thompson}
\end{algorithm}

Una consecuencia importante de este cambio en la representación es que el espacio de búsqeda se reduce a solo las soluciones activas y como se sabe que las planifaciones óptimas son un subconjunto de estas, esta representación puede representar cualquier solución óptima. 

Otro punto negativo es que esta representación es n a 1 porque solo importa el tamaño relativo de las llaves que compiten entre sí, es decir que diferentes valores de llaves pueden llevarnos a la misma solución. En principio no parece un problema muy grave auqnue sería más eficiente tener una representación 1 a 1 aunque se requiere de más análisis para plantearla.
