\section{Búsqueda local iterada}
Se propone utilizar la metaheurística de búsqueda local iterada por su simplicidad. Como se muestra en el algoritmo \ref{alg:ILS} la búsqueda local iterada requiere que definamos una manera de perturbar una solución dada así como definir un criterio de paro. La perturbación debe ser suficientemente grande como para permitirnos escapar de un óptimo local pero no tan grande como para eliminar toda la estructura que se tiene hasta el momento.

Por estas razones y para evitar complicar más el algoritmo con la definición de operadores completamente nuevos que realicen cambios arbitrarios, la perturbación implementada consiste simplemente en reemplazar a la solución actual por un vecino suyo (de acuerdo con la definición de vecindad que se esté usando). Esta definición es conveniente porque presenta una manera de aceptar cambios que no mejoran la solución pero que a su vez pueden estar conectados a soluciones de mejor calidad que la mejor actual.

 También es necesario establecer un criterio de paro, en este caso fue el tiempo, se tomaron 5 minutos lo cual representa una cantidad de tiempo muy pequeña comparada con la requerida por los métodos del estado del arte.

