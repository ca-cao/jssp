\section{Búsqueda local iterada}

Dado que uno de los objetivos de esta tesis es utilizar técnicas algorítmicas relativamente simples, y sólo introducir cierta complejidad
a través de la modificación del paisaje de búsqueda, se decidió utilizar la búsqueda local iterada.
%
Actualmente el paisaje de búsqueda de la mayor parte de los métodos del estado del arte consiste en la representación directa basada en permutaciones, la vecindad N7 y el makespan como función de fitness. Este paisaje de búsqueda se toma entonces como la base para proponer modificaciones a cada una de sus componentes.
%
Como se mostró anteriormente en el algoritmo~\ref{alg:ILS}, la búsqueda local iterada requiere que definamos una manera de perturbar una 
solución dada así como definir un criterio de paro. 
%
La perturbación debe ser suficientemente grande como para permitirnos escapar de un óptimo local pero no tan grande como para eliminar toda 
la estructura que se tiene hasta el momento.
%
Con el fin de evitar complicar el algoritmo con la definición de operadores completamente nuevos que realicen cambios arbitrarios, la 
perturbación implementada consiste simplemente en reemplazar a la solución actual por un vecino suyo (de acuerdo con la definición de vecindad 
que se esté usando). 
%
Este vecino es aceptado, independientemente de su nivel de calidad.
%
Así, esta definición es conveniente porque presenta una manera de aceptar cambios que no mejoran la solución pero que a su vez pueden estar 
conectados a soluciones de mejor calidad que la mejor actual.
%
Nótese que esta perturbación puede parecer muy pequeña, ya que habitualmente se incorporan perturbaciones bastante más disruptivas.
%
Sin embargo, esta tesis está enfocada en obtener soluciones de calidad relativamente alta en tiempo muy cortos.
%
Algunos estudios iniciales con operadores más disruptivos no ofrecieron ventajas sobre la propuesta de moverse simplemente a un vecino,
por lo que por simplicidad y eficiencia se decidió mantener esta decisión de diseño.

Por otro lado también es necesario establecer un criterio de paro. 
%
Teniendo en cuenta los objetivos de esta tesis, se fijó el criterio de paro a 5 minutos, lo cual representa una cantidad de tiempo muy pequeña comparada 
con la requerida por los métodos del estado del arte, que están del orden de varios días de ejecución.

