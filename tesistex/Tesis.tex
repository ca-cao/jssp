\documentclass[12pt,letterpaper]{book}
\usepackage[spanish, es-tabla]{babel}
\usepackage[utf8]{inputenc}
\usepackage[colorlinks=true, linkcolor=ginda]{hyperref}
\usepackage{booktabs}
\usepackage{amsmath}
\usepackage{graphicx}
\usepackage{amsfonts}
\usepackage{amssymb}
\usepackage{pdfpages}
\usepackage{makeidx}
\usepackage{import}
\usepackage{listings}
\usepackage{float} 
\usepackage{color}
\usepackage{multirow}
\usepackage{amsfonts}
\usepackage{minitoc}
\usepackage{enumitem}
\usepackage{setspace}
%\usepackage{apacite}
%\usepackage{natbib}
\usepackage[titletoc]{appendix}
%\usepackage{algorithm,algpseudocode}
\usepackage[ruled,vlined]{algorithm2e}
%\usepackage{algorithmicx}
\usepackage[Lenny]{fncychap}
\usepackage{subcaption}
\mtcsettitle{minitoc}{Sumario}
\mtcsetrules{*}{off}
\DeclareMathOperator*{\argmax}{arg\,max}
\DeclareMathOperator*{\argmin}{argmin}
\newcommand{\EA}{{\sc ae}}
\newcommand{\DE}{{\sc ed}}
\newcommand{\TS}{{\sc bt}}
\newcommand{\MA}{{\sc am}}
\newcommand{\GCP}{{\sc gcp}}
\newcommand{\KGCP}{k-{\sc gcp}}
\newcommand{\EAS}{{\sc ae}s}
\newcommand{\BNP}{{\sc bnp}}
\newcommand{\CEC}{{\sc cec}}
\newcommand{\PSO}{{\sc pso}}
\newcommand{\BFS}{{\sc bfs}}
\newcommand{\NSGA}{{\sc nsga-ii}}
\newcommand{\DECD}{{\sc decd}}
\newcommand{\BNPC}{{\sc bnpc}}
\newcommand{\OCR}{{\sc ocr}}
\newcommand{\DMACOL}{{\sc dmacol}}
\newcommand{\IEEE}{{\sc ieee}}
\newcommand{\DIMACS}{{\sc dimacs}}
\newcommand{\GECCO} {{\sc gecco}}
\newcommand{\TSP} {{\sc tsp}}
\newcommand{\GPP} {{\sc gpp}}

\newcommand{\SADE}{{\sc sade}}
\newcommand{\SHADE}{{\sc shade}}
\newcommand{\LSHADE}{{\sc l-shade}}
\newcommand{\LSHADECC}{{\sc l-shade44}}
\newcommand{\LSHADEE}{{\sc l-shade44-iepsilon}}
\newcommand{\LSHADEI}{{\sc l-shade44-ide}}
\newcommand{\CODE}{{\sc code}}
\newcommand{\CCODE}{{\sc c$^2$ode}}
\newcommand{\JADE}{{\sc jade}}
\newcommand{\UDE}{{\sc ude}}
\newcommand{\IUDE}{{\sc iude}}
\newcommand{\MAES}{{\sc ma-es}}

\newcommand{\COP}{{\sc cop}s}
\newcommand{\CO}{{\sc co}}
\newcommand{\DEP}{{\sc dep}}
\newcommand{\DEI}{{\sc de}}
\newcommand{\EAI}{{\sc ea}}
\newcommand{\EASI}{{\sc ea}s}

\newcommand{\DEBNP}{{\sc debnp}}
\newcommand{\DSATUR}{{\sc dsatur}}
\newcommand{\GPX}{{\sc gpx}}
\newcommand{\TABUCOL}{{\sc tabucol}}
\newcommand{\HBFSX}{{\sc hbfsx}}
\newcommand{\MBA}{{\sc mba}}
\newcommand{\FAP}{{\sc fap}}

\newcommand{\subsubsubsection}[1]{\paragraph{#1}\mbox{}\\}
\setcounter{secnumdepth}{2}
\setcounter{tocdepth}{2}

\newcommand{\R}{\mathbb{R}}
\newcommand\tab[1][1cm]{\hspace*{#1}}

%\algdef{SE}[DOWHILE]{Do}{doWhile}{\algorithmicdo}[1]{\algorithmicwhile\ #1}%
%\algrenewcommand\algorithmicprocedure{\textbf{procedimiento}}
%
%\algrenewcommand\algorithmicrequire{\textbf{Entrada:}}
%\algrenewcommand\algorithmicensure{\textbf{Salida:}}
%\algrenewcommand\algorithmicend{\textbf{fin}}
%\algrenewcommand\algorithmicif{\textbf{si}}
%\algrenewcommand\algorithmicthen{\textbf{entonces}}
%\algrenewcommand\algorithmicelse{\textbf{si no}}
%\algrenewcommand\algorithmicfor{\textbf{para}}
%\algrenewcommand\algorithmicforall{\textbf{para todo}}
%\algrenewcommand\algorithmicdo{\textbf{hacer}}
%\algrenewcommand\algorithmicwhile{\textbf{mientras}}
%\algrenewcommand\algorithmicloop{\textbf{repetir}}
%\algrenewcommand\algorithmicrepeat{\textbf{repetir}}
%\algrenewcommand\algorithmicreturn{\textbf{regresar}}
%\definecolor{ginda}{rgb}{0.52, 0.0, 0.13} 
\definecolor{ginda}{rgb}{0.52, 0.0, 0.13}
%\makeatletter \renewcommand{\ALG@name}{Algoritmo} \makeatother



\usepackage[left=2cm,right=2cm,top=2cm,bottom=2cm]{geometry}
\author{Juan Germán Caltzontzin Rabell}
\title{Paisaje de búsqueda en el problema de planificación de producción tipo taller}

\begin{document}

\includepdf[pages=1-1,
    picturecommand*={%
         \put(80,610){\Large{\textrm{\textbf{PAISAJE DE BÚSQUEDA EN EL PROBLEMA DE}}}}%
        \put(75,580){\Large{\textrm{\textbf{PLANIFICACIÓN DE PRODUCCIÓN TIPO TALLER}}}}%
         \put(260,500){\Large{\textbf{T E S I S}}}
         \put(210,470){\large{Que para obtener el grado de}}%
         \put(215,440){\Large{\textbf{Maestro en Ciencias}}}
         \put(235,410){\large{con Especialidad en}}%
         \put(135,380){\Large{\textbf{Computaci\'on y Matem\'aticas Industriales}}}
         \put(265,320){\large{\textbf{Presenta:}}}
         \put(185,290){\Large{Juan Germán Caltzontzin Rabell}}%
         \put(230,230){\large{\textbf{Director de Tesis:}}}
         \put(195,200){\Large{Dr. Carlos Segura González}}%
    %\put(212,238){\large{\textbf{Directores de Tesis:}}}
    %\put(218,212){\large{Dr. director de tesis 1}}%
    %\put(218,190){\large{Dr. director de tesis 2}}
         \put(200,120){\Large{--------------------------------}}
         \put(210,110){\Large{\textbf{Autorizaci\'on de la versi\'on final}}}
         \put(310,30){\large{Guanajuato, Gto., 11 de Noviembre de 2021}}
    }]{FondoPortada.pdf}


%\begin{minipage}{0.18\textwidth}
%	\includegraphics[width=0.8\textwidth]{Imagenes/cimatlogo.png}
%\end{minipage}%
%\begin{minipage}{0.82\textwidth}
%\begin{center}
%	\large \sc Centro de Investigaci{\'o}n en Matemáticas, A.C.
%\end{center}
%\end{minipage}
%\begin{center}
%\Large \bf Paisaje de búsqueda en el problema de planificación de producción tipo taller
%\end{center}
%
%\begin{center}
%{\LARGE  Tesis que presenta}\\ \vspace{0.5cm}
%{\LARGE \bf Juan Germán Caltzontzin Rabell} \\ \vspace{1cm}
%{\LARGE  para obtener el Grado de}\\ \vspace{0.5cm}
%\LARGE \bf Maestro en Ciencias con Especialidad en Computación y Matemáticas Industriales
%\end{center}
%
%\centerline{\LARGE  Director de Tesis}
%\vspace{0.5cm}
%\centerline{\LARGE \bf  Dr. Carlos Segura González }
% 
%\vspace{1.2cm}
%{\large \bf \hfill Guanajuato, Gto. Julio de 2021}
%
%\pagenumbering{gobble}% Remove page numbers (and reset to 1)
%\pagebreak
%
%\clearpage  \clearpage \clearpage

\pagebreak \frontmatter
\chapter*{Resumen}
Existe una gran cantidad de tipos de problemas de planificación, siendo el problema de planificación de producción tipo taller
(JSP por sus siglas en inglés) un tipo específico que ha atraído considerable atención por su dificultad y aplicabilidad. 
%
En general estos problemas consisten en encontrar una planificación que cumpla un conjunto de restricciones y optimice alguna función
objetivo, siendo la minimización del tiempo requerido para completar un conjunto de trabajos (makespan) una de las funciones objetivo más habituales.
%
En esta tesis se trabaja con la formulación estándar del JSP, en la que precisamente se debe minimizar el makespan, y en particular se analiza 
el efecto sobre el rendimiento de distintas estrategias de modificación del paisaje de búsqueda.

En la actualidad existen algoritmos exactos para encontrar la solución óptima del JSP~\cite{Brucker1994}.
%
Sin embargo, el tiempo requerido para resolver instancias grandes es demasiado elevado como para que sean prácticos o inclusive factibles.
%
La complejidad de este problema lo hizo un candidato ideal para utilizar métodos aproximados con el fin de generar soluciones aceptables 
en tiempos razonables. 
%
Así, en la actualidad, los mejores resultados obtenidos para las instancias de prueba más difíciles se han encontrado mediante metaheurísticas, y
en específico la metaheurística de trayectoria búsqueda tabú es una de las más efectivas.
%
Las metaheurísticas son convenientes por su relativa simplicidad algorítmica frente a otros métodos; sin embargo, el tiempo requerido para obtener 
soluciones de alta calidad también ha comenzado a crecer y el diseño de las técnicas se ha complicado enormemente, incluyendo así numerosas
componentes que no son para nada sencillas de programar e integrar entre sí.
%
A modo de ejemplo, algunos trabajos recientes consideran ejecuciones de varios días en ambientes paralelos, y que integran múltiples componentes
como algoritmos evolutivos, búsqueda tabú y modelos subrogados.
%
Estos requerimientos pueden no ser muy prácticos a la hora de resolver problemas de este tipo en empresas, por lo que esta tesis parte de la
idea de que es importante dar un paso atrás y replantear un poco la forma de aplicar estos métodos para evitar complicarlos en exceso.

En el campo de las metaheurísticas hay varios conceptos fundamentales que tienen un gran impacto en su desempeño.
%
En este trabajo se analizan tres de ellos que juntos componen lo que se conoce como paisaje de búsqueda: la representación de las soluciones, 
la definición de vecindad de una solución, y la función de aptitud o fitness. 
%
En el trabajo se proponen y analizan diferentes alternativas para cada una de estas componentes, y se evalúa el impacto sobre el rendimiento utilizando
para ello una de las metaheurísticas de trayectoria más sencillas, la búsqueda local iterada.
%
Esta evaluación experimental considera las instancias dmu, las cuales son uno de los conjuntos de prueba más utilizados actualmente y que aun no ha
sido resuelto por completo.
%
A partir de los análisis realizados se pudo concluir que de las tres componentes en las que se plantearon modificaciones, la que más impacto tiene para el
rendimiento es la representación de soluciones.
%
Entre otras propuestas, se diseñó una nueva representación basada en algunas propiedades que se estudiaron en los 90s, y junto con definiciones
de funciones de fitness y vecindades adecuadas, permitió mejorar de forma significativa la calidad de las soluciones encontradas en ejecuciones a corto plazo. 
%
Dado que esta representación difiere de las usadas en algoritmos mucho más complejos como búsqueda tabú, los resultados discutidos en esta tesis
abren la puerta a nuevas investigaciones que utilicen este tipo de representación más elaborada, junto a metaheurísticas de
trayectoria más complejas.
%
En esta tesis también se proponen nuevas funciones de fitness y definiciones de vecindad, mostrándose que, aunque permiten mejorar aún más las soluciones,
el efecto es mucho más limitado.
%
Entre las contribuciones, cabe destacar que se dispone de un optimizador, en el que en ejecuciones de 5 minutos en entornos secuenciales, 
la mediana del error relativos es menor al $0.21$ en todas las instancias tratadas, valor que es significativamente inferior en comparación de los casos 
en que no se incorporan los cambios propuestos en esta tesis, que alcanzan errores relativos de $0.34$.


\let\cleardoublepage\clearpage



 \newpage
\chapter*{Abstract}
En este trabajo se analiza el efecto de distintas metodologías de modificación del paisaje de búsqueda del problema de planificación de producción tipo taller (JSSP por sus siglas en inglés) (En inglés)

\let\cleardoublepage\clearpage

\dominitoc \tableofcontents \let\cleardoublepage\clearpage

\spacing{1.2}
\mainmatter

\chapter{Introducción}\label{cap:introduccion}  \minitoc
En este capítulo se presenta el problema de interés así como una revisión de los distintos métodos que se han formulado para resolverlo. Se plantea la hipótesis del trabajo y los objetivos así como las propuestas que servirán para alcanzar los objetivos. Al final del mismo se presentan las contribuciones de este trabajo al problema considerado.
\section{Antecedentes y Motivación}
% un pequeño review
Los problemas de planificación surgen de manera natural en sistemas de producción o procesamiento en los que la tarea general que se quiere completar consiste a su vez de varias subtareas que pueden o deben repartirse entre distintas máquinas o unidades de procesamiento. A grandes rasgos, un problema de planificación consiste en asignar a cada subtarea una máquina que debe procesarla y un tiempo de inicio y fin.  Estos problemas pueden surgir de todo tipo de contextos, desde la producción de algo como una bicicleta hasta la forma en que los sistemas computacionales procesan información o cómo se asignan las clases en una escuela. En muchos de estos contextos no solo se requiere hallar una planificación sino que a demás se quiere encontrar una que sea óptima en algún sentido, habitualmente se quiere la que haga que se complete el trabajo lo más rápido posible aunque puede haber otros criterios.\\
Desde los década de los 50 se comenzaron a formular algoritmos de planificación\cite{johnson1954optimal} y el interés en ellos ha crecido en las décadas siguientes hasta la fecha.\\ 
 
En este trabajo se trata un tipo especial de problema de planificación conocido como el \textbf{Problema de Planificación de Producción tipo Taller} o JSP por sus siglas en inglés. Este problema consiste en hallar una secuencia de procesamiento de $n$ items en $m$ máquinas que tome el mínimo tiempo posible cumpliendo ciertas restricciones de orden para el procesamiento de los $n$ items. Este es un problema de optimización combinatoria en la que se requiere encontrar una secuencia de procesamiento para las operaciones en cada una de las máquinas disponibles y es NP duro cuando el número de máquinas es mayor a dos.\\

Anteriormente se han propuesto métodos exactos para resolver este problema \cite{Brucker1994} pero la cantidad de recursos computacionales que requieren conforme el tamaño del problema (número de máquinas y trabajos) aumenta los hace imprácticos excepto para instancias pequeñas.
% poner explicacion de instancias


Dado el interés que se tiene en este problema y en que su tamaño puede ser suficientemente grande para no poder utlizar los métodos exactos, se han propuesto otros métodos aproximados para encontrar soluciones buenas en tiempos aceptables. Estos métodos pueden agruparse de la siguiente manera\cite{Zhang2019}:
\begin{itemize}
\item \textbf{Métodos Constructivos} Estos métodos construyen una planificación mediante el uso de una regla simple por lo que son muy rápidos y conceptualmente sencillos. En general pueden clasificarse en tres tipos: los que usan reglas de prioridad, los que usan heurísticas de cuello de botella y los que usan algún método de inserción. El primero de estos consiste en establecer una forma de elegir la operación a planificar de varias disponibles. Esto puede hacerse por ejemplo eligiendo la que tome más tiempo o la que pueda procesarse antes. El segundo consiste en replantear el problema como una serie de subproblemas más sencillos que puedan resolverse iterativamente hasta que se tenga una solución completa. El tercer tipo construye una solución partiendo del ordenamiento de solo un subconjunto de operaciones y progresivamente agregando más a partir de las que ya se tienen.
\item \textbf{Métodos de inteligencia artificial} En estos métodos utilizan redes neuronales para encontrar la planificación. Existen muchos tipos de redes que se han diseñado para atacar este problema aunque en general suelen combinarse con otros métodos para obtener resultados competitivos.
\item \textbf{Métodos de búsqueda local} En estos métodos se establece una forma de crear soluciones nuevas a partir de una solución dada para reemplazarla por una nueva y repetir el proceso hasta que se cumpla algún criterio de paro. Para crear nuevas soluciones se hace un cambio pequeño como, por ejemplo, intercambiar el orden de dos operaciones de modo que sea posible evaluar todas las posibles soluciones generadas al aplicar esta modificación. En general estos métodos se distinguen entre sí por la manera en que se escoge la solución con la cual reemplazar la solución inicial y el criterio de paro.
\item \textbf{Metaheurísticas} Estos métodos combinan elementos de los previamente mencionados para plantear estrategias que obtengan aun mejores resultados. Es muy común que se planteen metaheurísticas inspiradas en la naturaleza como los algoritmos genéticos basados en la evolución, algoritmos basados en el comportamiento de seres vivos o bien en procesos naturales como el recocido simulado. Al ser de alto nivel pueden adaptarse a una gran cantidad de problemas.
\end{itemize}

Actualmente los algoritmos más exitosos para resolver el JSP son algoritmos meméticos que combinan un método de búsqueda local con una metaheurística poblacional que mantiene varias soluciones. El método de búsqueda local más usado se conoce como búsqueda tabú que mediante el uso de memoria logra escapar de óptimos locales.

Aunque se han obtenido buenos resultados con estos algoritmos se requieren de 24 a 48 horas paralelas de ejecución. La literatura reciente se ha centrado en hacer más eficiente la búsqueda tabú para reducir los tiempos de ejecución, dejando de lado los otros elementos que forman parte de las metaheurísticas. 

Como se explicará a detalle más adelante el desempeño de las metaheurísticas depende del llamado paisaje de búsqueda del problema, el cual se compone de tres elementos.

El primero y más básico es cómo representar computacionalmente una solución, actualmente lo más común es que una solución se represente como un grafo dirigido cuyos nodos son las operaciones a procesar y las aristas indican el ordenamiento de éstas.

El segundo es la forma de generar una solución a partir de otra, esto se conoce como una estructura de vecindad y la más exitosa es conocida como N7 y fue propuesta en 2006 \cite{Zhang2007}. 

El tercer elemento se conoce como función de fitness o aptitud y nos permite comparar soluciones y decidir si una es mejor que otra. 



\section{Objetivo}
% hablar del tipo de modificaciones 
% utilizar métodos sencillos 

El objetivo de este trabajo es plantear modificaciones que permitan el uso de mecanismos más simples y rápidos para encontrar soluciones comparables al estado del arte.

% poner objetivos específicos
% fitness
% representación
% vecindad


\section{Hipótesis}

La hipótesis de este trabajo es que es posible mejorar de forma significativa los resultados obtenidos con optimizadores de trayectoria simples, 
si se redefine de forma adecuada el paisaje de búsqueda, y que en particular es posible conseguir resultados no muy lejanos de los reportados 
por los mejores algoritmos actuales, con algoritmos sencillos y rápidos.

\section{Propuestas}
\subsection*{Hipótesis}
Es posible conseguir resultados comparables al estado del arte con algoritmos sencillos y rápidos si modificamos el paisaje de búsqueda de manera adecuada.\\
Como se explica mas adelante, el paisaje de búsqueda es la combinación de tres elementos: función de fitness, espacio de búsqueda y estructura de vecindad. Estos tres elementos tienen un efecto muy importante en el éxito que puede llegar a tener una metaheurística
Para poner a prueba la hipótesis se presentan las siguientes propuestas que serán exploradas mediante experimentos computacionales:
\begin{enumerate}
\item Utilizar la búsqueda local iterada (ILS por sus siglas en inglés) por ser un algoritmo simple y rápido.
\item Agregar nuevos movimientos a la vecindad N7.
\item Crear una función de fitness que no solo tome en  cuenta el makespan sino también otras características de la solución.
\item Utilizar una nueva representación junto con un esquema de decodificación para limitar el espacio de búsqueda.
\item Construir una nueva estructura de vecindad a partir de la nueva representación.
\end{enumerate}


% poner constribuciones diciendo lo que se consiguió 
% lo más importante el significado de los nodos
% las demas no son tan significativas
% calidad en relación conotros de similar complejidad
% considerar cuando hay tiempos muy cortos y considerar el beneficio al etodo complejo
\section{Contribuciones}
En este trabajo se consiguió plantear modificaciones al paisaje de búsqueda que, combinados con la metaheurística búsqueda local iterada, obtuvieron resultados 
con una mediana de error relativo no mayor al $0.21$ para todas las instancias de prueba en un tiempo de 5 minutos no paralelo comparado con las 24 a 48 horas 
de cómputo paralelo necesarios para hallar dichas soluciones.
%
Además, estas mejoras fueron muy significativas en comparación con las casos en que no se incorporan dichas modificaciones, ya que en estos casos el error
relativo se incrementó hasta $0.34$ en las instancias más difíciles.

De todos los cambios que se implementaron al paisaje de búsqueda, el que tuvo más impacto en el desempeño de la búsqueda local iterada consistió en cambiar 
la representación con el fin de generar exclusivamente soluciones que cumplen ciertas propiedades que se saben que aparecen en soluciones óptimas.
%
No se logró plantear una función de fitness que mostrara una mejora sustancial en los resultados aunque definitivamente tiene un impacto positivo considerar 
algo más que únicamente el makespan de una planificación para hallar mejores soluciones.

Las modificaciones propuestas en esta tesis pueden servir como punto de partida para proponer estrategias más complejas que puedan acercarse mucho más a los 
resultados del estado del arte pero en una fracción del tiempo requerido actualmente. 
%
Por su puesto, conseguir esto es un reto importante que requiere refinar aún más la representación y la estructura de vecindad aquí presentadas, así como
estudiar la forma en que se debe integrar con muchas de las componentes que han llevado a encontrar las mejores soluciones conocidas hasta la fecha.





%\input{Introduccion/Panorama}
\newpage

\chapter{Marco teórico}\label{cap:marcot} \minitoc
% apartado optimizacion (metaheuristicas ...) 
% apartado definiciones (problema) 
% apartado union (vecindades ...)
En este capítulo se presentan diversos conceptos utilizados en este trabajo de tesis, así como definiciones relativas al problema 
y algunos algortimos específicos relevantes para el presente trabajo.

\section{Optimización}
La optimización es un área que tiene como propósito encontrar \textit{la mejor} entre diferentes opciones elegibles. 
%
En nuestra vida diaria a menudo nos encontramos en este tipo de situaciones, por ejemplo al elegir entre diferentes rutas para 
llegar a algún lugar, o al elegir entre diferentes productos con diversas características.

En lenguaje matemático un problema de optimización consiste en minimizar o maximizar una función que le asigna un valor a cada una 
de las opciones que tenemos disponibles. 
%
Para la definición formal se adopta la siguiente notación:
\begin{itemize}
    \item $X$ El conjunto de opciones o soluciones disponibles.
    \item $f:X\rightarrow \mathbb{R}$ La función objetivo a minimizar o maximizar.
\end{itemize}

El problema consiste en hallar:
\begin{gather}
\min_{x\in X} f(x)
\end{gather}

Es importante mencionar que cualquier problema en el que se requiera encontrar el argumento que hace tomar el valor máximo a $f$ 
puede transformarse en un problema equivalente de minimización con el reemplazo $f(x) \leftarrow -f(x)$, por lo que puede considerarse 
solo el caso de minimización sin pérdida de generalidad.

En función de las características del problema y método de optimización, estos se pueden clasificar de diferentes formas. 
%
A continuación se presentan algunas de las agrupaciones más importantes.

\subsection*{Optimización con restricciones}
A la definición de un problema de optimización también pueden agregarse un conjunto de restricciones, definidas habitualmente como
un conjunto de funciones que toman una solución como entrada una solución e indican si esta cumple o no con ellas, e incluso el grado
de incumplimiento.
%
Se dice que una solución es factible si cumple con todas las restricciones e infactible en caso contrario.
%
Estas restricciones se agregan ya sea porque el problema así lo requiere o bien para asegurarse que la solución tenga sentido, por ejemplo 
si $f$ tiene como argumento un número real que representa una longitud debe requerirse que esta longitud sea un número positivo.
%
En estos casos, el problema se convierte en encontrar una solución factible que optimice la función objetivo.

\subsection*{Optimización global y local}

Hallar el mínimo de la función sobre todo el conjunto $X$ recibe el nombre de optimización global, esto puede llegar a ser muy costoso computacionalmente,
por lo que es común optar por obtener simplemente un mínimo local, es decir, una solución que sea la mejor de un subconjunto de $X$. 
%
A menudo este subconjunto se define utilizando alguna medida de distancia entre soluciones y considerando a todas las soluciones que estén en algún rango 
de distancia a otra de referencia o bien añadiendo un operador que permita crear soluciones nuevas a partir de una solución ya conocida.

\subsection*{Optimización continua y discreta}
En la definición de problema de optimización dado anteriormente no se requiere que el conjunto de soluciones tenga alguna propiedad o alguna estructura 
adicional. 
%
Por ejemplo el conjunto de soluciones puede ser un conjunto numerable ( e.g. los enteros ) o no numerable ( e.g. los números reales ). 
%
Estos dos casos dividen a la optimización en dos ramas: optimización continua y optimización discreta. 
%
En general los problemas de optimización continua suelen ser más fáciles de abordar~\cite{nocedal2006numerical} en el sentido de que se puede utilizar
el concepto de gradiente para obtener información del valor de la función objetivo de puntos cercanos a cierto punto conocido mientras que en los problemas 
discretos esto rara vez puede hacerse.
%
Sin embargo, el uso de gradientes también tiene muchas limitaciones y dificultades, por lo que ambos tipos de optimización son muy complejos y siguen
siendo tema de investigación.

\subsection*{Optimización combinatoria}
Dentro de los problemas de optimización discreta se distinguen los problemas de optimización combinatoria, en los cuales el conjunto de búsqueda es finito.
%
Los problemas de optimización combinatoria, como el tratado en esta tesis, se pueden modelar con los siguientes elementos~\cite{Blum2003}:

\begin{itemize}
    \item Un conjunto de variables $Z=\{z_1,z_2,...,z_n\}$
    \item Dominio finito para cada variable $D_1,D_2,...,D_n$
    \item Restricciones entre variables
\end{itemize}

Estos tipo de problemas aparecen en múltiples escenarios, por ejemplo cuando se requiere hallar una permutación de objetos que optimice cierta función objetivo.

\subsection*{Métodos de optimización estocástica}

En relación a los métodos de optimización, esta tesis está enfocada en los métodos estocásticos.
%
En contraste con los métodos de optimización deterministas, los métodos estocásticos pueden obtener soluciones diferentes en cada ejecución aunque tenga las mismas 
condiciones iniciales ya que internamente toman decisiones de forma probabilista.
%
En muchas ocasiones es ventajoso tener esta aleatoriedad porque nos permite explorar el espacio de búsqueda con menos sesgos, lo que es especialmente útil cuando el 
espacio de búsqueda es poco <<predecible>> en el sentido en que no podemos a priori establecer reglas que nos permitan generar soluciones que estén en regiones de alta
calidad. 
%
Uno de los método estocásticos más conocidos es recocido simulado, el cual se inspira en un fenómeno metalúrgico y que encuentra soluciones muy buenas a una gran 
variedad de problemas de optimización.
%
Posteriormente, se describen otros métodos estocásticos, incluyendo a la búsqueda local iterada, siendo este último el método que se ha utilizado en esta tesis para
validar las propuestas realizadas.


\section{Teoría de la complejidad}
De manera empírica sabemos que existen problemas <<difíciles>> y problemas <<fáciles>>, por ejemplo, es mucho más difícil armar un rompecabezas que comprobar que está bien armado. La teoría de la complejidad busca responder a la pregunta: \textit{¿Qué hace a algunos problemas fáciles y a otros difíciles?} \cite{sipser1996introduction}. 

Una de los logros de la teoría de la complejidad ha sido establecer un sistema de clasificación de acuerdo con la dificultad de los problemas. Este sistema consiste en muchas clases a las cuales puede ser asignado un problema en relación a los recursos computacionales necesarios para resolverlo, siendo las clases más estudiadas las que se definen por cantidad de operaciones (tiempo) y por memoria (espacio). 
A continuación se explica brevemente cómo se procede para describir la cantidad de operaciones necesarias para resolver un problema.

\subsection*{Notación gran $O$}
La cantidad de operaciones necesarias para resolver algún problema puede expresarse como una función del tamaño del mismo, es decir de la forma $f(n)$ donde $n$ es el tamaño del problema. La función $f$ puede ser de infinidad de formas diferentes, pero para fines prácticos solo se considera su comportamiento asintótico, es decir, cuando $n$ tiende a infinito. De esta manera se puede tener un conjunto de clases de equivalencia en las que se considera que los problemas tienen el mismo nivel de dificultad. Para delimitar estas clases se define la notación gran $O$.

Para dos funciones $f,g:\mathbb{R}\rightarrow\mathbb{R}$ se dice que $f(n)= O(g(n))$\footnote{El signo de igual aquí no tiene el sentido usual sino que más bien representa una relación entre conjuntos en la que $=$ quiere decir $\subseteq$ \cite{graham1989concrete}} si existen constantes $c,n_0\in\mathbb{R}^+$ tal que $0\leq f(n)\leq cg(n)$ para todo $n\geq n_0$ \cite{cormen2009introduction}.

De manera intuitiva $f(n)=O(g(n))$ quiere decir que $cg(n)$ es una cota superior a $f(n)$ como puede verse en la figura \ref{fig:bigo}.
\begin{figure}[H]
    \centering
    \includegraphics[scale=.8]{Imagenes/bigo.png}
    \caption{Representación de $f(n)= O(g(n))$}
    \label{fig:bigo}
\end{figure}
Se dice que un algoritmo es $O(g(n))$ si el número de operaciones que requiere para resolver un problema de tamaño $n$ es $O(g(n))$. Por ejemplo, el algoritmo para sumar dos números de $n$ toma $O(n)$ operaciones. 

\subsection*{Clases \textbf{P} y \textbf{NP}}
La forma de la función $g$ es la base para clasificar los problemas de acuerdo a su complejidad dos de las clases más analizadas son definidas por el número de operaciones y se conocen como:
\begin{itemize} 
    \item \textbf{P} Los problemas para los que se conoce un algoritmo para resolverlos que toma a lo más $Kn^c$ operaciones con $K,c$ finitas. El nombre de esta clase hace referencia que pueden resolverse en tiempo polinomial es decir que toman $O(n^c)$ operaciones para alguna $c$ positiva y finita.
    \item \textbf{NP} Los problemas para los que se puede verificar que se encontró una solución en tiempo polinomial.
\end{itemize}

Claramente \textbf{P} $\subseteq$ \textbf{NP} pero determinar si también se cumple esta relación en sentido contrario, es decir determinar si \textbf{P} = \textbf{NP} es un problema sumamente difícil con implicaciones importantes que no parece estar cerca de ser resuelto.

La distinción en algoritmos que toman tiempo polinomial parece arbitraria aunque de manera empírica se conocen muchos problemas de interés que pueden ser clasifcados con esta definición. De un punto de vista más formal, los polinomios ejemplifican funciones de crecimiento lento y cumplen con varias propiedades teóricas que simplifican la clasificación de los problemas\cite{wigderson2006p}.

Muchos problemas de optimización de gran interés pertenecen a la clase \textbf{NP}. Ante este problema surgieron técnicas conocidas como metaheurísticas que buscan facilitar encontrar soluciones aceptables aunque no óptimas.




\section{Metaheurísticas}
Es muy común que en nuestra cotidianidad nos enfrentemos a problemas tan difíciles o para los que tengamos tan poco tiempo de decisión que no podamos hacer 
un análisis riguroso;
en estos casos es muy común que utilicemos algún método (posiblemente basado en la experiencia) que nos permita hallar una solución aceptable, por ejemplo, 
es común que reemplacemos el problema por uno más simple que sí podemos responder y cuya respuesta está relacionada con nuestro problema original.
%
Así, no podemos predecir con certeza si lloverá durante el día, pero sí podemos responder si el cielo está plagado de nubes oscuras.

En el contexto de la optimización, una metaheurística es una metodología de alto nivel que combina diferentes heurísticas y puede aplicarse para resolver 
de manera aproximada una gran cantidad de problemas. 
%
En la práctica existen numerosas metaheurísticas que pueden ser muy diferentes entre sí por lo que no hay un sistema de clasificación universalmente aceptado, 
aunque se han propuesto diferentes criterios de clasificación~\cite{Stegherr2020} con base en diferentes características:

\begin{itemize}
\item De trayectoria vs discontinua. Una metaheurística de trayectoria consiste en, dada una solución inicial, mejorarla de manera iterativa mediante algún operador que <<mueve>> a la solución a través del espacio de búsqueda de modo que sigue un camino o trayectoria en el espacio de búsqueda. En las discontinuas los movimientos no trazan una 
\item Basadas en población vs basadas en una sola solución. En las metaheurísticas basadas en población se mantiene un conjunto de soluciones candidatas que a menudo se combinan entre sí en lugar de modificar una sola.
\item Basadas en búsqueda local vs constructivas. Como se explicará más adelante, en la búsqueda local, el proceso de mejora implica la evaluación de soluciones muy parecidas a una solución inicial dada mientras que en las constructivas se crean nuevas soluciones de acuerdo a una heurística o algoritmo preestablecido.
\item Con uso de memoria vs sin uso de memoria. El uso de memoria consiste en almacenar información que nos ayude a explorar el espacio de búsqueda, por ejemplo una lista de soluciones previamente visitadas.
\end{itemize} 
Los primeros dos de estos criterios están muy relacionados porque casi todas las metaheurísticas discontinuas son poblacionales y muchas de trayectoria son basadas en una sola solución.


Si bien las metaheurísticas son muy diversas, existen elementos comunes que tienen un papel determinante en el buen funcionamiento de las mismas. 
%
En específico para las metaheurísticas de trayectoria, aunque también aplican para las demás, existen tres conceptos que determinan el llamado paisaje de búsqueda: 
la representación de las soluciones, la forma en que las soluciones están conectadas y cómo podemos compararlas entre sí. 
%
En la siguiente sección se introduce el concepto de paisaje de búsqueda y sus componentes, el cual es el principal tema de estudio de esta tesis.

\section{Paisaje de búsqueda}
El concepto de paisaje de búsqueda surgió en la biología para explicar los mecanismos de evolución de poblaciones de seres vivos~\cite{wright1932roles}. 
%
La idea básica es que los genes de algún organismo en particular le confieren ventajas o desventajas en su hábitat, i.e. puede estar más o menos adaptado. 
%
De esta manera puede pensarse en clasificar a todos los individuos de acuerdo a su grada de adaptación. 

Conforme se introducen nuevas variaciones en el código genético ya sea por la reproducción o mutación, se pueden añadir a los nuevos individuos a la clasificación. 
%
Si se continua de esta forma se obtendrá una especie de mapa en el cual podríamos asignar a cada código genético un punto en el mapa y un valor de adaptabilidad.
%
En algunas extensiones de este concepto también se agrega información sobre la procedencia de cada genoma.

El proceso previamente descrito tiene como resultado una estructura que puede representarse mediante un grafo. 
%
En el ámbito de las metaheurísticas el concepto de paisaje de búsqueda se inspira fuertemente en la descripción biológica anterior. 
%
Los conceptos de genoma, adaptabilidad y reproducción y mutación tienen como análogos la representación, la función de fitness o aptitud y la estructura de vecindad.   

\subsection{Representación}

En ocasiones, el problema de optimización en el que estemos interesados puede surgir directamente de un ámbito de las matemáticas;
%
sin embargo, si estamos interesados en un problema de nuestro entorno físico es necesario idear una forma de traducirlo a un lenguaje matemático, 
incluidas la soluciones al mismo. 
%
Por ello, debemos encontrar una forma de representar de manera útil las soluciones posibles.
%
Por ejemplo si buscamos un ordenamiento óptimo para algún conjunto de $n$ cosas podemos asignarle a cada elemento del conjunto un número entero del $1$ al $n$, 
en este modelo el espacio de soluciones está constituido por todas las permutaciones de los números del $1$ al $n$. 

Hay varias maneras de representar permutaciones, podemos usar un arreglo de $n$ entradas o bien una matriz de permutación por mencionar algunos. 
%
Algo sumamente importante es que estas dos formas de representar las soluciones son muy distintas y se tiene que trabajar con ellas de manera muy diferente. 
%
Por ejemplo, es notorio que la primera de ellas puede llegar a representar \textit{soluciones no factibles}, por ejemplo, si aparece un número repetido, mientras que 
la segunda no.

Con el ejemplo anterior también podemos ver que si queremos establecer algunos operadores que, por ejemplo, perturben la solución tendremos que definirlos de maneras 
completamente distintas, en función de la representación que estemos considerando.
%
También es importante notar que es posible que el número de soluciones factibles e infactibles cambian en función de la representación, por lo que dependiendo de
la misma, se puede facilitar o dificultar la búsqueda.

Formalmente una representación está formada por con conjunto de objetos (representaciones de soluciones) y un mapa que asocia elementos entre el conjunto de representaciones
y el conjunto de las soluciones. 
%
Este mapeo no tiene por que ser suryectivo, es decir, que puede ser que solo asigne una representación a parte del conjunto de soluciones. 
%
También puede ser el caso como se mencionó anteriormente que haya representaciones que no correspondan a soluciones. 
%
En función del tipo de mapeo entre representaciones y soluciones se pueden distinguir las siguientes~\cite{Cheng1996}:
\begin{itemize}
    \item $1$ a $1$ A cada solución le corresponde una única representación.
    \item $1$ a $n$ La misma representación puede asociarse a varias soluciones.
    \item $n$ a $1$ Una solución puede tener diferentes representaciones.
\end{itemize}

Nótese que en una representación se pueden combinar las propiedades anteriores, así una solución podría estar representada por una única representación
y otra por muchas.
%
Estas características también son importantes y afectan enormemente al rendimiento de la búsqueda.

\begin{figure}[H]
    \centering
    \includegraphics[scale=1]{Imagenes/representacion.pdf}
    \caption{Tipos de representaciones}
\end{figure}

En general, el mapeo $1$ a $1$ es el más utiilizado, sin embargo, todos estos tipos de mapeos han sido útiles.
%
En particular, si conseguimos un mapeo en el que las soluciones de más calidad, estén sobrerepresentadas, se pueden conseguir búsqueda más efectivas.

\subsection{Vecindad}
La definición de vecindad es crucial para las metaheurísticas de trayectoria y las basadas en una sola solución.
%
Formalmente, una vecindad es un mapeo $N:X\rightarrow 2^X$ que le asigna a cada solución $x\in X$ un subconjunto de soluciones en $X$. 
%
Intuitivamente podemos pensar que es una forma de definir a las soluciones que <<rodean>> a otra. 
%
Se dice que la solución $y$ es un vecino de $x$ si $y\in N(x)$.

A partir de la definición de vecindad podemos también definir un operador de movimiento $M:X\rightarrow X$ cuyo efecto al aplicarlo a una solución sea 
transformarla en una que pertenezca a su vecindad, i.e. este operador selecciona a un vecino de la solución inicial.  
\[M(x)=y\in N(x)\quad x\neq y\]

En general se busca que una vecindad cumpla con ciertas características, entre las más importantes es que las soluciones formen una componente conexa, 
que no tenga un tamaño excesivamente grande y que contenga las vecinos de una solución tenga una calidad no muy distinta a la de la solución inicial;
esta última característica le confiere <<suavidad>> y puede ser de ayuda para tener búsqueda más efectivas, ya que en caso contrario, evaluar una
solución prácticamente no ofrece información sobre la calidad de la región en la que se está buscando.

\subsection{Funci\'on de aptitud o fitness}
En el momento en el que se define el problema de optimización, se define cuál es la función objetivo; 
no obstante, puede resultar útil construir otra función a la cual se le conoce como función de aptitud o fitness que toma en cuenta otras características 
de las soluciones para que el paisaje de búsqueda tenga una estructura más favorable para los métodos de solución que se pretendan usar. 
%
Por ejemplo puede suceder que aunque dos soluciones tengan asociado el mismo valor de la función objetivo una de ellas posea características que la hacen 
un mejor punto de partida para alguna metaheurística.
%
Por ello, el diseño de funciones de aptitud o fitness apropiadas es muy importante para el buen desempeño de los algoritmos.

La función de aptitud debe asociar a cada solución un elemento de un conjunto en que se tiene definida una noción de ordenamiento. 
%
En esencia esta función define un operador de comparación entre soluciones de modo que define qué soluciones se consideran mejor que otras.
%
Una vez que tenemos el conjunto de soluciones representables y operadores de cambio para generar nuevas soluciones a partir de otras, se define 
el espacio de búsqueda como un grafo dirigido $G$ en el que los nodos son las soluciones al problema y una solución $x$ está conectada a otra $y$ 
si podemos generar a $y$ aplicando los operadores de cambio a $x$.

Además, podemos asociar a cada solución en el espacio un valor de aptitud o fitness que mide la calidad de dicha solución. 
%
La adición de esta función de aptitud al espacio de búsqueda genera al paisaje de búsqueda. 
%
Formalmente el paisaje de búsqueda $\mathcal{L}$ es entonces una tupla conformada por el espacio de búsqueda junto con una función objetivo que 
guía la búsqueda $\mathcal{L}=(G,f)$.

En la figura \ref{fig:landscape} se muestra una representación pictórica del paisaje de búsqueda para un problema de optimización.

\begin{figure}
\begin{subfigure}{.4\textwidth}
    \includegraphics[scale=.5]{Imagenes/search1.png}
    \caption{Soluciones representables}
\end{subfigure}
\begin{subfigure}{.5\textwidth}
    \includegraphics[scale=.5]{Imagenes/search2.png}
    \caption{Relaciones inducidas por los operadores de cambio}
\end{subfigure}
\begin{subfigure}{\textwidth}
    \centering
    \includegraphics[scale=.5]{Imagenes/search3.png}
    \caption{Adición de la función de fitness}    
\end{subfigure}
\caption{Creación del paisaje de búsqueda}
\label{fig:landscape}
\end{figure}

\medskip
El paisaje de búsqueda es el <<terreno>> a explorar y puede cambiar si cambiamos cualquiera de sus componentes.
%
Así, podría ser que alguna representación nos centre en un subconjunto de soluciones convenientes o que se proponga alguna estructura de vecindad 
de modo que las mejores soluciones nunca están muy lejos del resto, o que la función de fitness nos ayude a atravesar cúmulos de soluciones que serían iguales si se usara
directamente la función objetivo..

La estructura del paisaje de búsqueda influye de manera determinante en el éxito o fracaso de las metaheurísticas. 
%
Dependiendo de la <<forma>> del paisaje se favorecerá el uso de ciertas metaheurísticas. 
%
La <<forma>> del paisaje hace referencia a cómo cambia el valor de fitness para soluciones conectadas entre sí y para modelarlo de forma adecuada se usan diferentes
tipos de mediciones.
%
Una medida ampliamente usada estima cómo cambia el valor de la función de fitness conforme nos acercamos a óptimos locales, mientras que otra se gran popularidad 
estima qué tanto cambia el fitness entre soluciones vecinas\cite{skauffman}. 
%
La primera de estas medidas nos da una idea de qué tan grandes son los valles que rodean a un mínimo local si es que existen y la segunda nos da una ida de la 
rugosidad del paisaje.
%
En esta tesis no nos centramos en cómo cuantificar y analizar el paisaje de búsqueda, por lo que no se entra en más detalle.
%
El objetivo se centra en proponer modificaciones a los paisajes de búsqueda más usado para el JSP, con el fin de mejorar el rendimiento de metaheurísticas sencillas.

% imagen
\begin{figure}[H]
\begin{subfigure}{.45\textwidth}
    \includegraphics[scale=.4]{Imagenes/rugged.png}
    \caption{Paisaje rugoso con muchos óptimos locales}
\end{subfigure}
\begin{subfigure}{.5\textwidth}
    \includegraphics[scale=.4]{Imagenes/ruggedvalley.png}
    \caption{Paisaje rugoso con con un gran valle}
\end{subfigure}
\begin{subfigure}{\textwidth}
    \centering
    \includegraphics[scale=.4]{Imagenes/smoothvalley.png}
    \caption{Paisaje suave con un gran valle}
\end{subfigure}
\caption{Diferentes tipos de paisajes de búsqueda. Se muestran paisajes continuos con fines ilustrativos.}
    \label{fig:landtypes}
\end{figure}

\subsection*{Algunas metaheurísticas de relevancia}
Las metaheurísticas sirven como una estrategia para explorar el paisaje de búsqueda. 
%
Una de las más sencillas e intuitivas es conocida como escalada estocástica y simplemente consiste en reemplazar la solución actual por algún vecino mejor 
escogido al azar hasta que la solución en la que estemos sea mejor que todos sus vecinos, es decir, un óptimo local. 
%
Esta es una metaheurística de trayectoria y traza un camino entre las soluciones inicial y final.

%
\begin{algorithm}[H]
 \KwData{Problema de Optimización}
    \KwResult{Óptimo local $x$}
 Generar solución inicial $x$\;
 \While{$L$ no vacía}{
    Generar lista de vecinos $L$ de $x$\;
    Escoger al azar un vecino $y\in L$\;
  \eIf{$y<x$}{
      $x \leftarrow y$\;
      Generar lista de vecinos $L$ de $x$\;
   }{
       Quitar a $y$ de $L$\;
  }
 }
    \Return{x}
    \label{alg:LS}
    \caption{Algoritmo de escalada estocástica}
\end{algorithm}

\begin{figure}[H]
\centering
\includegraphics[scale=.7]{Imagenes/mettray.png}
    \caption{Ilustración de una escalada estocástica. Las aristas rayadas representan la vecindad del óptimo local.}
\end{figure}


% introducir ejemplos de mh de trayectoria
Otra metaheurística de trayectoria importante que se ha considerado en la mayor parte de métodos que han alcanzado las mejores soluciones conocidas para el JSP es
la llamada búsqueda tabú. 
%
Esta metaheurística se basa en la búsqueda local pero puede aceptar vecinos que no mejoran la función de fitness con la intención de evitar atascarse en un óptimo 
local de mala calidad. 
%
La búsqueda tabú almacena soluciones previamente vistas o características de las mismas para evitar visitarlas de nuevo. 
%
De entre las soluciones vecinas no prohibidas por la lista tabú, se mueve a la mejor.
%
Es necesario definir el tamaño de la lista, el tipo de información que contendrá y el criterio de paro.

\begin{algorithm}[H]
 \KwData{Problema de Optimización}
 \KwResult{Mejor solución encontrada $x^*$}
 Generar solución inicial $x$\;
    Inicializar mejor solución como $x^*\leftarrow x$
 Inicializar la lista tabú $TL$ con $x$\; 
 \While{no criterio de paro}{
     Generar lista de vecinos aceptables $L$ de $x$ tal que $L\cap TL =\varnothing$\;
    Escoger un vecino $y\in L$\;
      $x \leftarrow y$\;
      Actualizar $TL$\;
      \If{$x<x^*$}{
      $x^* \leftarrow x$\;
      }
 }
    \Return{$x^*$}
    \label{alg:TS}
    \caption{Algoritmo básico de búsqueda tabú}
\end{algorithm}

\smallskip
Por último se presenta una metaheurística de trayectoria muy sencilla que intenta subsanar el problema de atascarse en óptimos locales de mala calidad de la búsqueda local,
la búsqueda local iterada.
%
La idea de esta metaheurística es obtener un mínimo local a partir de una solución inicial mediante búsqueda local, y aplicarle una perturbación seguida de búsqueda local para
evitar el óptimo local.
%
Este proceso se repite hasta que se cumpla algún criterio de paro. 
%
Esta estrategia es conceptualmente muy simple aunque se debe definir la perturbación que se hace a la solución y por lo general no es tan sencillo plantear una perturbación adecuada.\\

\begin{algorithm}[H]
 \KwData{Problema de Optimización}
 \KwResult{Mejor solución encontrada $x^*$}
 Generar solución inicial $x$\;
 Inicializar mejor solución como $x^*\leftarrow x$
 \While{no criterio de paro}{
     Obtener una solución $y$ a partir de $x$ mediante búsqueda local \ref{alg:LS}\;
     \eIf{$y<x^*$}{
     $x \leftarrow y$\;
     $x^* \leftarrow y$\;
    }{
     $x \leftarrow x^*$\;
     Perturbar $x$\;
    }
 }
    \Return{$x^*$}
    \label{alg:ILS}
    \caption{Algoritmo búsqueda local iterada}
\end{algorithm}

\section{Problema de planificación de producción tipo taller (JSP)}
Una instancia del JSP consiste en $n$ trabajos diferentes constituidos cada uno por $m$ operaciones que deben procesarse por un tiempo determinado en $m$ máquinas en una secuencia predeterminada.\\
El objetivo es hallar la planificación que minimiza el tiempo que toma terminar todos los trabajos dado que cada máquina puede procesar solo un trabajo a la vez.

Una planificación consiste en asignar tiempos de inicio y fin a cada operación respetando el orden requerido para cada trabajo. El tiempo que toma terminar todos los trabajos se conoce como makespan y la secuencia de trabajos que toma el mayor tiempo en completarse se conoce como ruta crítica. La ruta crítica puede verse como una serie de bloques críticos que consisten en las secuencias de operaciones de la ruta crítica que se ejecutan de forma adyacente en la misma máquina. Una planificación puede tener una o varias rutas críticas. \\
Sin pérdida de generalidad podemos restringirnos al caso en el que el tiempo requerido para procesar cada operación es un entero positivo.

\subsection*{Ejemplo}
Se muestra un ejemplo de una instancia con 3 máquinas y 2 trabajos.
\begin{table}[H]
\centering
\caption{Instancia simple con 3 maquinas y 2 trabajos}
\begin{tabular}{@{}llll@{}}
Trabajo & \multicolumn{3}{l}{\begin{tabular}[c]{@{}l@{}}Secuencia de procesamiento \\ (máquina, tiempo)\end{tabular}} \\ \midrule
0       & 0, 75                              & 2, 54                               & 1, 59                             \\ \midrule
1       & 0, 47                              & 2, 72                              & 1, 45   \\\hline                         
\end{tabular}
\label{tab:inst}
\end{table}

La siguiente es una posible planificación para la instancia de ejemplo, visualizada mediante un diagrama de gantt. En negro se marca los trabajos que conforman la ruta crítica. 
\begin{figure}[H]
\centering
\includegraphics[scale=.7]{Imagenes/planejemplorc.png}
\caption{Diagrama de gantt de una planificación posible}
\label{fig:gantt}
\end{figure}

Si bien es conveniente visualizar una planificación mediante un diagrama como el mostrado anteriormente, resulta ventajoso representar estas planifaciones de otras formas. La representación de soluciones a un problema resulta ser una elección sumamente importante al momento de desarrollar, implementar y analizar los algoritmos que se propongan para resolver el problema. En la siguiente sección se desarrollan dos formas de representación de especial importancia.

\subsection{Representación de planificaciones}
De manera general existen dos formas de representar las soluciones al JSP \cite{Cheng1996}:
\begin{itemize}
    \item \textbf{Representación directa} Se almacena el orden de los trabajos en cada máquina o sus tiempos de inicio y final.
    \item \textbf{Representación indirecta} Se almacena información con la que se puede construir una planificación mediante un proceso de decodificación.
\end{itemize}
En este trabajo se utilizaron dos: el grafo disyuntivo (representación directa) y las reglas de prioridad (representación indirecta).
\subsubsection*{Modelo de grafo disyuntivo} 
En este modelo las planificaciones se representan con un grafo dirigido $G=(V,A,E)$ en el que $V$ es un conjunto de nodos que representa las operaciones, las aristas $A$ representan la secuencia que deben seguir las operaciones dentro de un mismo trabajo y $E$ es otro conjunto de aristas que indica el orden de procesamiento en cada una de las máquinas. Es importante mencionar que con este modelo podemos representar planificaciones no factibles, esto se da cuando el grafo $G$ contiene un ciclo.


Formalmente en una instancia del JSP se representa cada operación como un nodo, se agregan dos nodos de control que sirven como el nodo inicial (del que dependen todos los trabajos) y final (que depende de todos los trabajos), las restricciones de precedencia dentro de cada trabajo se representan como aristas dirigidas fijas llamadas aristas conjuntivas y las operaciones que deben procesarse en una misma máquina se unen mediante aristas llamadas disyuntivas, una solución factible o planificación se obtiene al elegir la dirección para cada arista disyuntiva de modo que no se generen ciclos.   
%El problema de hallar una planificación para cada una de las $m$ operaciones de los $n$ trabajos en las $m$ máquinas se reduce a elegir una permutación de los $n$ trabajos en cada máquina por lo que el número de posibles soluciones es $O(n!^m)$.

% poner las representaciones
\begin{figure}
    \centering
    \begin{subfigure}{.8\textwidth}
        \centering
        \includegraphics[width=.8\linewidth]{Imagenes/disyuntive.pdf}
        \caption{Representación de una instancia, los colores distinguen entre las tres máquinas.}
    \end{subfigure}
    \begin{subfigure}{.8\textwidth}
        \centering
        \includegraphics[width=.8\linewidth]{Imagenes/plandisyuntive.pdf}
        \caption{Planificación obtenida al fijar las aristas disyuntivas como en \ref{fig:gantt}. La ruta critica se resalta en amarillo}
    \end{subfigure}
\caption{Modelo de grafo disyuntivo para la instancia de ejemplo \ref{tab:inst}}
\end{figure}

\subsubsection*{Reglas de prioridad}
En esta representación una planificación se construye al aplicar un proceso de simulación en el que para cada maquina se construye una cola con las operaciones cuyas dependencias ya han sido procesadas. Inicialmente se tienen en las colas solo las operaciones iniciales de cada trabajo. Una vez que se tiene esto se utiliza una regla de prioridad para elegir qué operación debe planificarse en qué máquina. Se actualizan las colas para las máquinas que lo requieran y se continua con este proceso hasta completar la planificación (vaciar las colas).

Existen muchas reglas de prioridad que toman en cuenta cosas como la duración de la operación, la cantidad de operaciones restantes, la duración del trabajo al que pertenece una operación, entre muchas otras. La calidad de la planificación construida depende de la regla de prioridad que se utilice y de la estructura de la instancia en sí.


\subsection{Tipos de planificaciones}
Independientemente de cómo se representen o como se construyan las planificaciones pueden clasificarse en varios conjuntos. Dentro de el conjunto de planificaciones factibles se pueden distinguir dos subconjuntos de interés para el presente trabajo: el conjunto de las planificaciones óptimas que está conformado por las planificaciones con el menor makespan posible y el conjunto de las planificaciones activas. 
Estas últimas se definen como las planificaciones en las que no es posible disminuir el tiempo de inicio de ninguna operación sin aumentar el tiempo de inicio de otra. Es conocido que las planificaciones óptimas representan un subconjunto de las activas\cite{Ponsich2013}.

% imagen 

\begin{figure}[H]
    \centering
    \includegraphics[scale=.8]{Imagenes/solspace.pdf}
    \caption{Subconjuntos de planificaciones}
\end{figure}

Estas clasificaciones resultan interesantes porque existen algoritmos para generar tipos de planificaciones de modo que podamos centrarnos en solo un subconjunto del espacio de búsqueda.

%Como se juntan las definiciones con lo demas
\section{Metaheurísticas aplicadas al JSP}
La complejidad del JSP hace que las metaheurísticas sean actualmente los métodos más utilizados para resolver instancias grandes. Estas han conseguido hallar buenas soluciones para los conjuntos de prueba más populares. Aunque se han propuesto muchas metaheurísticas, práticamente todas utilizan el concepto de vecindad. Como ya se explicó, la definición de una estructura de vecindad es parte del paisaje de búsqueda y tiene un gran impacto en los resultados obtenidos. Desde que se propuso la primera vecindad en 1996, se han propuesto veecindades cada vez con mejores resultados.

\subsection*{Vecindades previamente propuestas}
Se han propuesto varias estructuras de vecindad al JSP, a continuación se describen las más importantes a la fecha:

\begin{itemize}
\item N1 \cite{blazewicz1996job} Consiste en considerar todas las soluciones que se crean al intercambiar cualquier par de operaciones adyacentes que pertenecen a un bloque crítico. Esta vecindad es muy grande y considera muchos cambios que no mejoran el makespan.
\begin{figure}[H]
\centering
\includegraphics[scale=.7]{Imagenes/N1.pdf}
\caption{Movimientos de la vecindad N1}
\end{figure}

\item N4 \cite{dell1993applying} Esta vecindad se propuso como un refinamiento y extensión de la vecindad N1 y toma como base el concepto de bloque crítico. Consiste en llevar operaciones internas del bloque crítico al inicio o final. 
\begin{figure}[H]
\centering
\includegraphics[scale=.7]{Imagenes/N4.pdf}
\caption{Movimientos de la vecindad N4}
\end{figure}


\item N5 \cite{EugeniuszNowicki2003} Consiste en intercambiar solo las operaciones adyacentes a la final o inicial de un bloque crítico.  
\begin{figure}[H]
\centering
\includegraphics[scale=.7]{Imagenes/N5.pdf}
\caption{Movimientos de la vecindad N5}
\end{figure}

\item N6 \cite{Balas1998} Los autores utilizan varios teoremas para identificar pares $(u,v)$ de operaciones dentro de un bloque crítico que puedan llevar a mejorar la solución y a su vez identificar si se tiene que mover a $u$ justo después de $v$(forward) o bien a $v$ justo antes de $u$ (backward).
\begin{figure}[H]
\centering
\includegraphics[scale=.7]{Imagenes/N6.pdf}
\caption{Los dos tipos de movimientos para un par $(u,v)$}
\end{figure}

\item N7 \cite{Zhang2007} Esta vecindad se plantea como una extensión de la N6 en la cual se toma la idea de los movimientos entre pares de operaciones de un bloque crítico. Los autores toman en cuenta todos los cambios posibles entre el inicio o fin del bloque crítico con todas las operaciones internas.

\begin{figure}[H]
\centering
\includegraphics[scale=.7]{Imagenes/N7.pdf}
\caption{Movimientos de la vecindad N7}
\end{figure}
\end{itemize}

Las vecindades antes presentadas se han centrado en operaciones que pertenecen a la ruta crítica por varias razones, la vecindad resultante es suficientemente pequeña como para ser explorada en su totalidad, el makespan de una planificación solo puede reducirse haciendo cambios en la ruta crítica y pueden garantizarse que cualquier vecino representa una solución factible.\\

La literatura reciente se ha centrado en hacer los algoritmos existentes más eficientes dejando de lado la parte de la representación o de la función de fitness. Existen otras ideas que no se han explorado a fondo pero parecen prometedoras como proponer extensiones a alguna de las vecindades, cambiar la representación y proponer una función de fitness que no solo tome en cuenta el makespan de modo que tengamos más formas de diferenciar entre soluciones. En la siguiente sección se presentan algunas otras medidas de calidad para planificaciones del JSP.
% ve como se pueden mejorar 
\subsection*{Criterios de optimalidad}
Como ya se ha mencionado el criterio de optimalidad más usado es el makespan, no obstante existen muchos otros criterios de optimalidad que pueden usarse para asignar un valor de fitness a una planificación. 

Si denotamos por $C_i$ al tiempo de finalización del trabajo $J_i$ y $f_i(C_i)$ a su costo asociado, podemos distinguir dos tipos de funciones de costo en la literatura\cite{Brucker2001}:
\[f_{\max}:=\max\{f_i(C_i)\}\]
y 
\[\sum f_i(C):=\sum_{1\leq i\leq n}f_i(C_i)\]

Los costos asociados a cada uno de los tiempos de finalización de los trabajos pueden tomar muchas formas, por ejemplo pueden introducirse pesos para cada trabajo o fijar tiempos de finalización esperado para cada trabajo y medir la desviación de ellos.

Dependiendo del problema en sí, puede ser que se le de más o menos valor a distintos aspectos de la planificación como el tiempo que están detenidas las máquinas, o el tiempo que tarda un trabajo en particular. El makespan se ha extendido como criterio de optimalidad porque está muy relacionado con los costos económicos de la planifacición\cite{Rand1977}.


Con las definiciones anteriores sobre vecindades, función objetivo y representaciones de las soluciones podemos construir el paisaje de búsqueda para el JSP. 
\section{Paisaje de búsqueda del JSP}

La estructura del paisaje de búsqueda influye de manera determiante en el éxito o fracaso de las metaheurísticas. Dependiendo de la <<forma>> que tenga el paisaje se favorecera el uso de ciertas metaheurísticas. La <<forma>> del paisaje hace referencia a cómo cambia el valor de fitness para soluciones conectadas entre sí. Dos medidas que son generalmente utlizadas para esto miden cómo cambia el valor de la función de fitness conforme nos acercamos a un óptimo local y la otra mide qué tanto cambia el fitness entre soluciones vecinas\cite{skauffman}. La primera de estas medidas nos da una idea de qué tan grandes son los valles que rodean a un mínimo local si es que existen y la segunda nos da una ida de la rugosidad del paisaje.

% imagen
\begin{figure}[H]
    \centering
    \subfigure[Paisaje rugoso con muchos óptimos locales]{\includegraphics[scale=.3]{Imagenes/rugged.png}}
    \subfigure[Paisaje rugoso con con un gran valle]{\includegraphics[scale=.3]{Imagenes/ruggedvalley.png}}
    \subfigure[Paisaje suave con un gran valle]{\includegraphics[scale=.3]{Imagenes/smoothvalley.png}}
    \caption{Diferentes tipos de paisajes de búsqueda. Se muestran paisajes continuos con fines ilustratios}
\end{figure}

En el caso del JSP se ha encontrado que el paisaje de búsqueda para una instancia al azar tiende a ser muy dependiente de la razón entre el número de máquinas y número de trabajos de la misma \cite{Streeter2006}. También se ha observado que el paisaje tiende a tener muchos óptimos locales de baja calidad para instancias dificiles \cite{mattfeld1999search} y que en general algunas soluciones están mucho mas conectadas que otras \cite{bierwirth2004landscape}. 

Estas características explican en parte por qué los métodos basados en búsqueda local como la búsqueda tabú combinados con métodos sofisticados de exploración han tenido tan buenos resultados. Puede ser que las soluciones que esten más conectadas a otras sean de baja calidad y sea necesario evitar ser <<atraidos>> hacia ellas para llegar a soluciones de mejor calidad.

En este sentido, podemos pensar que para que una metaheurística basada en búsqueda local tenga mayor probabilidad de encontrar buenas soluciones debemos plantear el paisaje de búsqueda de modo que no nos atasquemos en óptimos locales de mala calidad, ya sea porque están muy conectados y funcionan como atractores o bien porque el paisaje de búsqueda es tan rugoso que hay óptimos locales por doquier.

\newpage

\chapter{Propuestas}\label{cap:prop}
En este capítulo se detallan las modificaciones hechas a cada una de las componentes del paisaje de búsqueda.

\subsection*{Búsqueda local iterada}
Como se muestra en el algoritmo \ref{alg:ILS} la búsqueda local iterada requiere que definamas una manera de perturbar una solución dada así como definir un criterio de paro. La perturbación debe ser suficientemente grande como para permitirnos salir de un óptimo local pero no tan grande como para eliminar toda la estructura que se tiene hasta el momento.

Por estas razones y para evitar complicar más el algoritmo con la definición de operadores completamente nuevos que realicen cambios arbitrarios, la perturbación implementada consiste simplemente en reemplazar a la solución actual por un vecino suyo (de acuerdo con la definición de vecindad que se esté usando). Esta definicion es conveniente porque presenta una manera de aceptar cambios que no mejoran la solución pero que a su vez están conectados a soluciones de mejor calidad que la mejor actual.

 El criterio de paro en este caso fue el tiempo, se tomaron 5 minutos para todos los experimentos lo cual representa una cantidad de tiempo muy pequeña comparada con la requerida por los métodos del estado del arte.

El algoritmo se ejecuta 50 veces para cada instancia de modo que tenemos un conjunto de resultados que podemos comparar entre las distintas variantes propuestas al algoritmo.

\section{Grafo disyuntivo}
Inicialmente se trabajó con la representación del grafo disyuntivo \cite{balas1969machine}. Una solución factible se representa como un grafo dirigido acíclico en el que las aristas marcan el orden en el que se procesan las operaciones dentro de las máquinas. Esta representación se ha usado ampliamente en otros trabajos, siendo particularmente útil en aquellos que plantean métodos exactos de solución basados en enumeración completa\cite{Brucker2001}. 


Formalmente en una instancia del JSP se representa cada operación como un nodo, se agregan dos nodos de control que sirven como el nodo inicial (del que dependen todos los trabajos) y final (que depdende de todos los trabajos), las restricciones de precedencia dentro de cada trabajo se representan como aristas dirigidas fijas y las operaciones que deben procesarse en una misma máquina se unen mediante aristas disyuntivas, una solución factible o planificación se obtiene al elegir la dirección para cada arista disyuntiva de modo que no se generen ciclos.  

% figura 

\section{Llaves aleatorias}
La representación propuesta se basa en asignar a cada operación una número real entre 0 y 1 el cual sirve para definir un orden entre operaciones en una misma máquina mediante un proceso de decodificación un planteamiento similar puede encontrarse en \cite{bean1996}.

Para decodificar la solución a partir de las llaves para cada operación se utiliza el algoritmo de Giffler \& Thompson \cite{Giffler1960} con el cual se generan soluciones activas.
\section{Extensión a vecindad N7}
Como punto de partida se planteó agregar movimientos a la vecindad N7 con la cual se han obtenido los resultados del estado del arte.
Los movimientos que plantea esta vecindad solo tienen que ver con pares de operaciones en la ruta crítica por lo que una extensión sencilla consiste en considerar movimientos de operaciones que pueden no pertenecer a la ruta crítica. Es importante resaltar que si no se planteara también una función de fitness que no tome solo en cuenta el makespan estos movimientos podrían nunca llevarían a una mejora \cite{blazewicz1996job}. 
Los movimientos planteados se basan en observar que en alguna solución encontrada por una búsqueda local para cada maquina pueden existir periodos de tiempo en la que está inactiva pero existe alguna operación en la ruta critica que podría comenzar a procesarse en este periodo y que se procesa después. Ninguna de las vecindades previamente propuestas considera movimientos de las operaciones de la ruta crítica más allá del bloque crítico por lo que estos movimientos representan un conjunto previamente no explorado de soluciones.

Intuitivamente lo que se pretende es llenar un <<hueco>> en la planificación con una operación de un bloque crítico.

\begin{figure}[H]
\centering
\includegraphics[scale=.7]{Imagenes/N8.pdf}
    \caption{Movimientos propuestos. El tiempo de inicio de \textbf{v} es mayor al de finalización de \textbf{v}}
\end{figure}

\section{Vecindad basada en soluciones activas}
Esta vecindad surge del cambio de representación propuesto. En cada paso del algoritmo para construir una solución activa se consideran varias operaciones que <<compiten>> para ser planificadas (i.e. son planificables en ese momento) de las cuales se elige la que tiene la llave de mayor valor numérico. 

La idea es construir la vecindad a partir de estas operaciones que compiten. En un principio puede pensarse en considerar todos los posibles ordenamientos posibles para dichas operaciones pero esto da lugar a una vecindad demasiado grande por lo que se considera únicamente hacer cambios por pares de llaves entre la operación elegida y todas sus competidoras.
\begin{figure}[H]
\centering
\includegraphics[scale=1.3]{Imagenes/vec2.pdf}
\caption{Movimientos de la vecindad propuesta. $k(O^*)$ es la llave de la operación elegida}
\end{figure}
%diagrama

\section{Función de fitness}
La idea principal es la de tener un arreglo de características de la planificación y compararlos lexicográficamente para distinguir entre las soluciones. Para plantear qué características son las que se iban a tomar en cuenta para este arreglo se tomaron en cuenta otros criterios de optimalidad hallados en la literatura así como propuestas propias. También se planteó considerar todos los tiempos de finalización de las máquinas. Las características tomadas en cuenta fueron las siguientes:
\begin{itemize}
\item $C_{max}$ Makespan 
\item $\sum C_i^2$ Tiempo al cuadrado total 
\item $\sum J_i$ Flowtime 
\item $\sum I(C_i=C_{max})$ Número de máquinas cuyo tiempo de finalización es igual al makespan 
\item Número de rutas críticas 
\item $\sqrt{Var(C_i)}$ Desviación estándar de los tiempos de finalización 
\end{itemize}
Estas características se combinaron de varias formas como se mostrará en la validación experimental. 


\chapter{Validación experimental}
En este capítulo se detallan las modificaciones hechas a cada una de las componentes del paisaje de búsqueda.

\section{Conjuntos de instancias de prueba}
A lo largo de los años se han propuesto múltiples conjuntos de instancias de prueba para medir el desempeño de los algoritmos para la resolución del JSP. Uno de los más famosos fue propuesto en 1963~\cite{fisher1963} y para una de sus instancias de $10\times 10$ la solución óptima no fue encontrada sino hasta finales de los 80s~\cite{carlier1989}.\\
%
Debido al aumento de poder de cómputo y al desarrollo de mejores métodos de solución se han desarrollado conjuntos de prueba cada vez más desafiantes.
%
En la tabla \ref{tab:benchmark} se muestran en orden cronológico los conjuntos más usados. La mayor parte de ellos ya han sido resueltos casi en su totalidad.\\ 
%
\begin{table}[h]
\centering
\begin{tabular}{@{}cccc@{}}
\toprule
Nombre & \begin{tabular}[c]{@{}c@{}}Rango de tamaños\\ (trabajos$\times$máquinas)\end{tabular} & Número de instancias & \begin{tabular}[c]{@{}c@{}}Instancias sin\\ solución óptima\end{tabular} \\ \midrule
ft ~\cite{fisher1963} & $6\times 6$ - $20\times 5$     & 3  & 0  \\
la ~\cite{lawrence1984resource} & $10\times 5$ - $15\times 15$   & 40 & 0  \\
abz~\cite{adams1988shifting} & $10\times 10$ - $20\times 15$  & 5  & 1  \\
orb~\cite{applegate1991computational} & $10\times 10$                & 10 & 0  \\
swv~\cite{storer1992new} & $20\times 10$ - $50\times 10$  & 20 & 5  \\
yn ~\cite{yamada1992genetic} & $20\times 20$                & 4  & 3  \\
ta~\cite{taillard1993benchmarks}  & $15\times 15$ - $100\times 20$ & 80 & 21 \\ 
dmu~\cite{demirkol1998benchmarks} & $20\times 15$-$50\times 20$  & 80 & 54 \\ \midrule
\end{tabular}
    \caption{Conjuntos de instancias de prueba populares}
    \label{tab:benchmark}
\end{table}
%
En general, para generar un conjunto de instancias de prueba se elige un conjunto de valores para los tamaños de las instancias, una distribución de probabilidad para generar los tiempos de procesamiento de cada operación y se determina una manera de asignar las restricciones de precedencia en cada trabajo.
%
La dificultad de las instancias resultantes depende básicamente de dos factores: el tamaño de la instancia y la forma en la que se eligen las restricciones de precedencia en cada trabajo. Los tiempos de procesamiento de cada operación por lo regular se obtienen de una distribución uniforme.\\
%
Puede aumentarse arbitrariamente el tamaño de las instancias para hacerlas más difíciles aunque estos aumentos en dificultad no necesariamente se deben a que el problema sea fundamentalmente más complicado de resolver.
%
Existen varias formas de elegir las restricciones de precedencia. Como cada trabajo se procesa solamente una vez en cada una de las $m$ máquinas, escoger estas restricciones de precedencia equivale a escoger una permutación ($\sigma_i$) de las $m$ maquinas para cada trabajo. Puede escogerse al azar de entre todas las  permutaciones posibles, pero se ha observado que una forma de hacer una instancia mucho más desafiante es dividir a las máquinas en subconjuntos y escoger una permutación para cada uno de ellos. Esto genera un cuello de botella que aumenta mucho el makespan de las soluciones en general y que genera óptimos locales de muy baja calidad. \\

%
Por ejemplo si las máquinas se dividen en dos subconjuntos de igual tamaño, las restricciones de precedencia se eligen escogiendo una permutación de las primeras $k = \lfloor\frac{m}{2}\rfloor$ y concatenando una permutación de las restantes de la forma $\sigma_i = (\sigma_{i0}(0,1,\dots,k),\sigma_{i1}(k+1,k+2,m))$.\\
%
En la figura \ref{fig:bottleneck} podemos observar un caso simple de esto para 5 trabajos ($n=5$) y dos máquinas ($m=2$) en el que la operación inicial de cada trabajo debe procesarse en la misma máquina.\\
%
\begin{figure}[h]
\begin{subfigure}{\textwidth}
    \centering
    \includegraphics[scale=.5]{Imagenes/casoextremomalo.png}
    \caption{Planificación aleatoria}
\end{subfigure}
\begin{subfigure}{\textwidth}
    \centering
    \includegraphics[scale=.5]{Imagenes/casoextremobueno.png}
    \caption{Planificación óptima}
\end{subfigure}
\caption{Ejemplo de planificación para una instancia en la cuál todos los trabajos empiezan en la misma máquina}
\label{fig:bottleneck}
\end{figure}

%
Actualmente el conjunto más popular y reciente es el llamado dmu\cite{demirkol1998benchmarks} también conocido como \textbf{DMU01-80}. La segunda mitad de estas instancias \textbf{DMU40-80} son consideradas especialmente difíciles porque siguen el esquema antes mencionado en el que todos los trabajos tienen operaciones iniciales en la misma mitad de las máquinas. Este conjunto es el que se eligió para probar las propuestas presentadas en este trabajo.


% capitulo validacion experimental
\section{Organización de los resultados}
Como se explicó en los capítulos anteriores, las propuestas planteadas en este trabajo son de dos tipos: modificar el paisaje de búsqueda del problema y modificar la metaheurística con la que se explora este paisaje.
En el caso de la metaheurística la única modificación es usar la búsqueda local iterada (ILS). Para el caso del paisaje de búsqueda se presentan varias modificaciones a cada una de sus componentes: representación de soluciones, estructura de vecindad y función de fitness.

Los resultados a cada una de las componentes se muestran de modo que las modificaciones se van agregando para mejorar el desempeño del algoritmo. Primero se presenta el cambio en metaheurística seguido del cambio en la función de fitness posteriormente se considera la extensión de vecindad planteada y por último el cambio de representación junto con la nueva vecindad propuesta. Se asignan acrónimos para facilitar la comparación. El orden de presentación es el siguiente:
\begin{enumerate}
    \item \textbf{PN7Cmax} Representación basada en permutaciones, vecindad N7 y makespan como función de fitness.
    \item \textbf{PN7*} Representación basada en permutaciones, vecindad N7 y diversas funciones de fitness.
    \item \textbf{PN7extTup} Representación basada en permutaciones, vecindad N7 extendida y tupla ordenada de tiempos de finalización como función de fitness.
    \item \textbf{RpKeTup} Representación basada en llaves aleatorias, vecindad propuesta, y tupla ordenada de tiempos de finalización función de fitness.
\end{enumerate}

Cada una de las variaciones propuestas se ejecutó durante 5 minutos por 50 veces para cada instancia con el fin de obtener un conjunto de resultados que podemos comparar.

En las secciones siguientes se muestran los resultados de forma gráfica al comparar la mediana del error relativo respecto al mejor resultado reportado en la literatura, los cuales se muestran en el apéndice \ref{tab:sota}.
\section{Resultados para PN7Cmax}
Estos resultados sirven como una base para determinar si las modificaciones posteriores resultan en mejoras apreciables puesto que el único cambio con los métodos más extendidos en la literatura es el uso de la búsqueda local iterada.

Se muestran los resultados de manera gráfica para facilitar su visualización. Los resultados detallados se encuentran en el apéndice \ref{app:resn7ils}. 

\begin{figure}[H]
    \begin{subfigure}{\textwidth}
        \centering
        \includegraphics[scale=.65]{Imagenes/resn7ils1.png}
        \caption{Resultados para las instancias \textbf{DMU01-40}}
    \end{subfigure}
\end{figure}
\begin{figure}[H]\ContinuedFloat
    \begin{subfigure}{\textwidth}
        \centering
        \includegraphics[scale=.65]{Imagenes/resn7ils2.png}
        \caption{Resultados para las instancias \textbf{DMU41-80}}
    \end{subfigure}
    \caption{Resultados para la propuesta más simple. Se marcan los casos en los que se llegó a la mejor solución conocida.}
    \label{fig:PN7Cmax}
\end{figure}

En la figura \ref{fig:PN7Cmax} podemos observar que la mediana del error relativo para la segunda mitad de las instancias es considerablemente mayor que para la primera mitad en varios casos incluso llegando a igualar el mejor resultado conocido.

\section{Resultados para distintas funciones de fitness}
  La función de fitness se obtiene al construir la dupla formada por el makespan y alguna característca en ese orden. Para determinar cuál es la que obtiene mejores resultados las modificaciones se comparan a pares en cada instancia. Si se encuentra que la diferencia entre dos modificaciones es estadísticamente significativa, se le suma un punto a la ganadora y se le resta uno a la perdedora.
Para determinar si los conjuntos de resultados muestran diferencias estadísticamente significativas se utiliza la prueba de Wilcoxon con un nivel de significancia de $0.01$. 

Los resultados para las propuestas mostradas en \ref{prop:fitness} pueden verse de manera condensada en la figura\ref{fig:fcomp}. La casilla $(i,j)$ muestra el.\\
Todas las pruebas se realizaron con la vecindad N7 y la representación basada en permutaciones. Se muestra también una tupla construida con las características que obtuvieron mejores resultados presentada como \textbf{PN7C2/flow/VarC}. En la tabla \ref{tab:fcomp} se presentan ordenadas las variantes por el puntaje obtenido. Podemos observar que las primeras tres entradas están relativamente cerca aunque se ven diferencias claras entre ellas.\\

Podemos observar en la figura \ref{fig:fcomp} y en la tabla \ref{tab:fcomp} que el makespan fue la peor función de fitness. Para la tupla construida con las mejores características se observan pocas mejoras por lo que podemos concluir que agregar más características no lleva necesariamente a mejores resultados. Los mejores resultados se obtienen al construir la tupla de tiempos ordenados de finalización de las máquinas por lo que de ahora en adelante la función de fitness queda fija de esta manera y en los resultados subsecuentes solo se considera este caso. Los resultados detallados para esta función de fitness se muestran en el apéndice \ref{app:resn7tuple}.
Con estos resultados podemos concluir que es ventajoso enfocarse en mejorar las máquinas que tardan más tiempo siendo las tres mejores variantes las que toman más en cuenta estas máquinas.

\begin{figure}[h]
    \centering
    \includegraphics[scale=.8]{Imagenes/fitnesscomp.png}
    \captionof{figure}{Condensado de los resultados para las modificaciones a la función de fitness. }
    \label{fig:fcomp}
\end{figure}
\begin{table}[h]
    \centering
\begin{tabular}{@{}cc@{}}
Variante & Puntaje \\ \midrule
\toprule
    \textbf{PN7Tup} & 264.0 \\ \midrule
    \textbf{PN7C2/flow/VarC} & 257.0 \\ \midrule
    \textbf{PN7C2} & 252.0 \\ \midrule
    \textbf{PN7Flow} & 209.0 \\ \midrule
    \textbf{PN7VarC} & 142.0 \\ \midrule
    \textbf{PN7Rc} & -281.0 \\ \midrule
    \textbf{PN7Icmax} & -382.0 \\ \midrule
    \textbf{PN7Cmax} & -461.0 \\ \midrule
\end{tabular}
    \caption{Puntaje para cada una de las variantes de función de fitness propuesta}
    \label{tab:fcomp}
\end{table}

En la figura \ref{fig:PN7CmaxvsPN7Tup} se muestra una comparación de los resultados hallados para la mejor función de fitness \textbf{PN7Tup} contra el caso base que también fue el que obtuvo peores resultados \textbf{PN7Cmax}. Se pueden apreciar mejoras sustanciales con solo la adición de la función de fitness.
\begin{figure}[H]
    \begin{subfigure}{\textwidth}
        \centering
        \includegraphics[scale=.65]{Imagenes/PN7CmaxvsPN7Tup_1.png}
        \caption{Resultados para las instancias \textbf{DMU01-40}}
    \end{subfigure}
\end{figure}
\begin{figure}[H]\ContinuedFloat
    \begin{subfigure}{\textwidth}
        \centering
        \includegraphics[scale=.65]{Imagenes/PN7CmaxvsPN7Tup_2.png}
        \caption{Resultados para las instancias \textbf{DMU41-80}}
    \end{subfigure}
    \caption{Resultados para la propuesta más simple. Se marcan los casos en los que se llegó a la mejor solución conocida.}
    \label{fig:PN7CmaxvsPN7Tup}
\end{figure}


\section{PN7extTup}
Los resultados para la extensión que considera movimientos con espacios de inactividad de las máquinas se comparan con los mejores mostrados en la sección pasada \textbf{PN7Tup}. Se sigue el mismo procedimiento que en la sección anterior para hacer esta comparación. En la figura \ref{fig:n8vsn7} podemos ver que esta modificación no representa una mejora drástica en cuanto a los resultados anteriores.

\begin{figure}[H]
    \centering
    \includegraphics[scale=.7]{Imagenes/n8vsn7.png}
    \caption{Resultados de la comparación}
    \label{fig:n8vsn7}
\end{figure}

También se muestra de manera gráfica los resultados para todas las instancias junto con los mejores resultados de la sección anterior. Los resultados detallados para la extensión propuesta se encuentran en el apéndice \ref{app:resn8tuple}.

\begin{figure}[H]
    \begin{subfigure}{\textwidth}
        \centering
        \includegraphics[scale=.6]{Imagenes/n8vsn7err1.png}
        \caption{Resultados para las instancias \textbf{DMU01-40}}
    \end{subfigure}
\end{figure}
\begin{figure}[H]\ContinuedFloat
    \begin{subfigure}{\textwidth}
        \centering
        \includegraphics[scale=.6]{Imagenes/n8vsn7err2.png}
        \caption{Resultados para las instancias \textbf{DMU41-80}}
    \end{subfigure}
    \caption{Resultados para los métodos \textbf{PN7extTup} y \textbf{PN7tup}. Se marcan los casos en los que se llegó a la mejor solución conocida.}
\end{figure}

\section{Cambio de representación y vecindad}
El cambio de representación y vecindad se compara con las dos propuestas previas. Se sigue el mismo procedimiento mencionado para determinar si hay una diferencia significativa entre los resultados obtenidos.
\begin{figure}[H]
    \centering
    \includegraphics[scale=.7]{Imagenes/prn7n8comp.png}
    \caption{Resultados de la comparación entre los tres métodos.}
\end{figure}
También se muestra la mediana del error relativo alcanzado por los distintos métodos. Los resultados detallados para la extensión de vecindad se encuentran en el apéndice \ref{app:resprtuple}.

\begin{figure}[H]
    \begin{subfigure}{\textwidth}
        \centering
        %\includegraphics[height=.78\textwidth,width=.95\textheight,angle=270]{Imagenes/n8vsn7err1.png}
        \includegraphics[scale=.6]{Imagenes/prvsn7vsn8err1.png}
        \caption{Resultados para las instancias \textbf{DMU01-40}}
    \end{subfigure}
\end{figure}
\begin{figure}[H]\ContinuedFloat
    \begin{subfigure}{\textwidth}
        \centering
        %\includegraphics[height=.78\textwidth,width=.95\textheight,angle=270]{Imagenes/n8vsn7err2.png}
        \includegraphics[scale=.6]{Imagenes/prvsn7vsn8err2.png}
        \caption{Resultados para las instancias \textbf{DMU41-80}}
    \end{subfigure}
    \caption{Resultados para ambos métodos. Se marcan los casos en los que se llegó a la mejor solución conocida.}
\end{figure}


Para resaltar las diferencias entre los dos métodos se tomó la instancia en la que se obtuvieron los resultados más dispares, en este caso fue la \textbf{DMU78} y se registró para cada óptimo local visitado su makespan así como el tamaño de su vecindad.

\begin{figure}[H]
    \includegraphics[scale=.6]{Imagenes/compvec78.png}
    \caption{Comparación de tamaño de la vecindad contra makespan de los óptimos locales para la instancia \textbf{DMU78} }
    \label{fig:mattgraph}
\end{figure}

Otro modo de visualizar las diferencias entre ambas representaciones también se presentan diagramas de caja para la diferencia relativa relativa del makespan de los óptimos locales con sus vecinos. Esto nos da una idea de cómo se relacionan las soluciones de acuerdo con su makespan. Si estos valores son muy cercanos entre sí esto nos da un indicio de la suavidad del paisaje de búsqueda.
\begin{figure}[H]
    \begin{subfigure}{\textwidth}
        \centering
        %\includegraphics[height=.78\textwidth,width=.95\textheight,angle=270]{Imagenes/n8vsn7err1.png}
        \includegraphics[scale=.6]{Imagenes/bxpn7_1.png}
        \caption{Representación original con vecindad n7 instancias \textbf{DMU01-40}}
    \end{subfigure}
\end{figure}
\begin{figure}[H]\ContinuedFloat
    \begin{subfigure}{\textwidth}
        \centering
        %\includegraphics[height=.78\textwidth,width=.95\textheight,angle=270]{Imagenes/n8vsn7err2.png}
        \includegraphics[scale=.6]{Imagenes/bxppr_1.png}
        \caption{Representación propuesta instancias \textbf{DMU01-40}}
    \end{subfigure}
    \caption{Diagramas de caja de la diferencia relativa en makespan de un óptimo local con sus vecinos}
    \label{fig:bxp1}
\end{figure}

\begin{figure}[H]
    \begin{subfigure}{\textwidth}
        \centering
        %\includegraphics[height=.78\textwidth,width=.95\textheight,angle=270]{Imagenes/n8vsn7err1.png}
        \includegraphics[scale=.6]{Imagenes/bxpn7_2.png}
        \caption{Representación original con vecindad n7 instancias \textbf{DMU41-80}}
    \end{subfigure}
\end{figure}
\begin{figure}[H]\ContinuedFloat
    \begin{subfigure}{\textwidth}
        \centering
        %\includegraphics[height=.78\textwidth,width=.95\textheight,angle=270]{Imagenes/n8vsn7err2.png}
        \includegraphics[scale=.6]{Imagenes/bxppr_2.png}
        \caption{Representación propuesta instancias \textbf{DMU41-80}}
    \end{subfigure}
    \caption{Diagramas de caja de la diferencia relativa en makespan de un óptimo local con sus vecinos}
    \label{fig:bxp2}
\end{figure}

En las figuras \ref{fig:bxp1} y \ref{fig:bxp2} observamos que los vecinos de los óptimos locales para la nueva representación y vecindad son más parecidos.

% resultados


%Conclusiones 
\newpage
\chapter{Conclusiones y Trabajos a Futuro}\label{cap:ct} \minitoc
El problema de planificación tipo taller (JSP) es un problema de optimización combinatoria que ha recibido mucha atención desde hace varias décadas por sus aplicaciones 
en ambientes de manufactura. 
%
Este problema pertenece a la clase \textbf{NP-hard}, por lo que no se conocen algoritmos eficientes para resolverlo de manera general.
%
Es por ello que durante los últimos años, se han desarrollado múltiples optimizadores para el JSP basados en métodos aproximados. 
%
Dichos optimizadores han incrementando progresivamente su complejidad en implementación y mantenimiento, así como los requerimientos
de cómputo.

En la actualidad, los optimizadores que ofrecen los mejores resultados son técnicas híbridas que combinan múltiples ideas de muy diferente índole 
y requieren de grandes cantidades de recursos computacionales para su buen desempeño. 
%
En general el rendimiento de los optimizadores dependen del paisaje de búsqueda, es decir, de la representación que se le da a las soluciones, de la estructura de vecindad 
y de la función de fitness. 
%
Aunque se han usado diferentes tipos de representaciones, vecindades y funciones de fitness, gran parte de los esquemas actuales consideran aspectos comunes, y en particular, 
suelen usar la vecindad conocida como N7, una representación basada en permutaciones y el makespan como función de fitness. 
%
En cuanto a la forma de explorar el paisaje de búsqueda, actualmente los métodos más exitosos se basan en la metaheurística conocida como búsqueda tabú, pero incorporando 
múltiples modificaciones.

En esta tesis se propone un conjunto de modificaciones a cada una de las componentes del paisaje de búsqueda con el fin de analizar qué efecto tienen 
los mismos sobre el rendimiento de una metaheurística de trayectoria simple. 
%
En concreto, con el fin de obtener soluciones de alta calidad en tiempos reducidos, se usa una metaheurística simple: la búsqueda local iterada.
%
Entre las propuestas cabe destacar una manera de extender la vecindad N7 con movimientos que buscan aprovechar tiempos inactivos en la planificación,
nuevas funciones de fitness que permiten distinguir entre soluciones con el mismo makespan, una nueva representación basada en
llaves aleatorias a cada operación junto con un algoritmo para construir una solución a partir de ellas, que genera soluciones activas, y
una nueva estructura de vecindad asociada a esta última representación. 

Los resultados muestran que, para las instancias más difíciles de la actualidad, la utilización de un paisaje de búsqueda adecuado permite reducir
significativamente el error relativo, y en particular, se consiguió un optimizador que obtiene en menos de 5 minutos errores relativos no mayores a $0.21$.
%
Entre las modificaciones propuestas, la representación junto con la nueva estructura de vecindad, y las funciones de fitness fueron las que tuvieron mayor
efecto en los resultados.
%
La nueva representación es particularmente útil para la segunda mitad de las instancias DMU, las cuales son consideradas actualmente como las más retadoras.
%
Por otro lado, la función de fitness también es bastante importante.
%
En particular, debido a la existencia de múltiples zonas planas cuando solo se usa el makespan como función de fitness, el optimizador se queda bloqueado 
en óptimos de baja calidad. 
%
Sin embargo, al añadir características relevantes como lo son los tiempos de finalización de otras máquinas, surgen nuevas direcciones de mejora que permiten 
dirigir la búsqueda hacia óptimos locales de mejor calidad. 
%
Se probaron varias formas de extender la función de fitness y en particular, la que obtuvo mejor rendimiento consiste en construir una tupla con los tiempos 
ordenados de finalización de cada una de las máquinas y compararlas lexicográficamente para distinguir si una solución es mejor que otra. 
%
Esta función de fitness logró recudir la mediana del error relativo en todas las instancias y en varias de ellas se redujo aproximadamente a la mitad.
%
En lo que se refiere a la extensión de la vecindad N7 propuesta, no hay ventajas tan claras, siendo mejor en algunas instancias y significativamente peor en otras, por 
lo que se debe ver como una alternativa, en lugar de como una mejora.

Además de presentar los resultados, se realizaron algunos estudios para analizar los efectos que tienen los cambios introducidos en el paisaje de búsqueda.
%
En primer lugar, la nueva estructura de vecindad asociada a la representación de llaves, tiene una conectividad más homogénea, mientras que sin usar la misma,
las soluciones de menor makespan tienden a estar menos conectadas.
%
En segundo lugar, y probablemente más importante, la nueva vecindad muestra una mayor suavidad en el paisaje de búsqueda lo cual implica que las soluciones tienden 
a agruparse con soluciones de calidad similar, lo que se sabe que es muy bueno para el rendimiento de las metaheurísticas.
%

Nótese que en esta tesis, el paisaje de búsqueda sólo se integró con la metaheurística búsqueda local iterada, lo que permitió generar buenas soluciones en 
tiempos reducidos de solo 5 minutos.
%
Dadas las importantes mejoras obtenidas, se abre la puerta a incluir estos cambios con otras metaheurísticas de mayor complejidad y con metaheurísticas híbridas. 
%
Esto es un reto importante porque en su mayor parte, estas metaheurísticas incluyes pasos dependientes de la representación como cálculo aproximado de fitness, 
prohibiciones de ciertos tipos de movimientos, etc. por lo que no es para nada trivial realizar la integración, aunque parece muy prometedor.
%
Sería muy interesante plantear formas de aprovechar todo el desarrollo que se tiene para la representación basada en permutaciones y traducirlo a la representación 
propuesta en este trabajo o hacer que ambas representaciones trabajen en conjunto. 
%
Por último, la construcción de la función de fitness parece ser una gran área de oportunidad, y se podría incluir en metaheurísticas de mayor complejidad. 

\appendix
\chapter{Apéndice}
\section{Mejores resultados reportados en la literatura}
\begin{table}[H]
\centering
\begin{tabular}{@{}ccc@{}}
\toprule
Instancia & Tamaño & Estado del arte \\ \midrule
DMU01 & 20$\times$15 & 2563\\ 
DMU02 & 20$\times$15 & 2706\\ 
DMU03 & 20$\times$15 & 2731\\ 
DMU04 & 20$\times$15 & 2669\\ 
DMU05 & 20$\times$15 & 2749\\ 
DMU06 & 20$\times$20 & 3244\\ 
DMU07 & 20$\times$20 & 3046\\ 
DMU08 & 20$\times$20 & 3188\\ 
DMU09 & 20$\times$20 & 3092\\ 
DMU10 & 20$\times$20 & 2984\\ 
DMU11 & 30$\times$15 & 3430\\ 
DMU12 & 30$\times$15 & 3492\\ 
DMU13 & 30$\times$15 & 3681\\ 
DMU14 & 30$\times$15 & 3394\\ 
DMU15 & 30$\times$15 & 3343\\ 
DMU16 & 30$\times$20 & 3751\\ 
DMU17 & 30$\times$20 & 3814\\ 
DMU18 & 30$\times$20 & 3844\\ 
DMU19 & 30$\times$20 & 3764\\ 
DMU20 & 30$\times$20 & 3703\\ 
DMU21 & 40$\times$15 & 4380\\ 
DMU22 & 40$\times$15 & 4725\\ 
DMU23 & 40$\times$15 & 4668\\ 
DMU24 & 40$\times$15 & 4648\\ 
DMU25 & 40$\times$15 & 4164\\ 
DMU26 & 40$\times$20 & 4647\\ 
DMU27 & 40$\times$20 & 4848\\ 
DMU28 & 40$\times$20 & 4692\\ 
DMU29 & 40$\times$20 & 4691\\ 
DMU30 & 40$\times$20 & 4732\\ 
DMU31 & 50$\times$15 & 5640\\ 
DMU32 & 50$\times$15 & 5927\\ 
DMU33 & 50$\times$15 & 5728\\ 
DMU34 & 50$\times$15 & 5385\\ 
DMU35 & 50$\times$15 & 5635\\ 
DMU36 & 50$\times$20 & 5621\\ 
DMU37 & 50$\times$20 & 5851\\ 
DMU38 & 50$\times$20 & 5713\\ 
DMU39 & 50$\times$20 & 5747\\ 
DMU40 & 50$\times$20 & 5577\\ \bottomrule
\end{tabular}
\quad
\begin{tabular}{@{}ccc@{}}
\toprule
Instancia & Tamaño & Estado del arte \\ \midrule
DMU41 & 20$\times$15 & 3248\\ 
DMU42 & 20$\times$15 & 3390\\ 
DMU43 & 20$\times$15 & 3441\\ 
DMU44 & 20$\times$15 & 3475\\ 
DMU45 & 20$\times$15 & 3266\\ 
DMU46 & 20$\times$20 & 4035\\ 
DMU47 & 20$\times$20 & 3942\\ 
DMU48 & 20$\times$20 & 3763\\ 
DMU49 & 20$\times$20 & 3710\\ 
DMU50 & 20$\times$20 & 3729\\ 
DMU51 & 30$\times$15 & 4156\\ 
DMU52 & 30$\times$15 & 4303\\ 
DMU53 & 30$\times$15 & 4378\\ 
DMU54 & 30$\times$15 & 4361\\ 
DMU55 & 30$\times$15 & 4263\\ 
DMU56 & 30$\times$20 & 4941\\ 
DMU57 & 30$\times$20 & 4653\\ 
DMU58 & 30$\times$20 & 4701\\ 
DMU59 & 30$\times$20 & 4616\\ 
DMU60 & 30$\times$20 & 4721\\ 
DMU61 & 40$\times$15 & 5171\\ 
DMU62 & 40$\times$15 & 5248\\ 
DMU63 & 40$\times$15 & 5313\\ 
DMU64 & 40$\times$15 & 5226\\ 
DMU65 & 40$\times$15 & 5184\\ 
DMU66 & 40$\times$20 & 5701\\ 
DMU67 & 40$\times$20 & 5779\\ 
DMU68 & 40$\times$20 & 5763\\ 
DMU69 & 40$\times$20 & 5688\\ 
DMU70 & 40$\times$20 & 5868\\ 
DMU71 & 50$\times$15 & 6207\\ 
DMU72 & 50$\times$15 & 6463\\ 
DMU73 & 50$\times$15 & 6136\\ 
DMU74 & 50$\times$15 & 6196\\ 
DMU75 & 50$\times$15 & 6189\\ 
DMU76 & 50$\times$20 & 6718\\ 
DMU77 & 50$\times$20 & 6747\\ 
DMU78 & 50$\times$20 & 6755\\ 
DMU79 & 50$\times$20 & 6910\\ 
DMU80 & 50$\times$20 & 6634\\ \bottomrule
\end{tabular}
\caption{Mejores resultados reportados en la literatura a la fecha.}
\label{tab:sota}
\end{table}
\label{app:sotares} \newpage
\begin{table}[H]
\centering
\begin{tabular}{@{}ccccc@{}}
\toprule
\multirow{2}{*}{Instancia} & \multirow{2}{*}{Tamaño} & \multicolumn{2}{c}{ILS con N7} & \multirow{2}{*}{Estado del arte} \\ \cmidrule(lr){3-4}
& & Mejor& Mediana & \\ \midrule
DMU01 & 20$\times$15 & 2793 & 2867 & 2563\\ 
DMU02 & 20$\times$15 & 2853 & 3003 & 2706\\ 
DMU03 & 20$\times$15 & 2880 & 2968 & 2731\\ 
DMU04 & 20$\times$15 & 2831 & 2912 & 2669\\ 
DMU05 & 20$\times$15 & 2934 & 3056 & 2749\\ 
DMU06 & 20$\times$20 & 3395 & 3565 & 3244\\ 
DMU07 & 20$\times$20 & 3239 & 3359 & 3046\\ 
DMU08 & 20$\times$20 & 3332 & 3423 & 3188\\ 
DMU09 & 20$\times$20 & 3247 & 3363 & 3092\\ 
DMU10 & 20$\times$20 & 3100 & 3215 & 2984\\ 
DMU11 & 30$\times$15 & 3681 & 3819 & 3430\\ 
DMU12 & 30$\times$15 & 3737 & 3955 & 3492\\ 
DMU13 & 30$\times$15 & 3893 & 4082 & 3681\\ 
DMU14 & 30$\times$15 & 3526 & 3650 & 3394\\ 
DMU15 & 30$\times$15 & 3452 & 3548 & 3343\\ 
DMU16 & 30$\times$20 & 3985 & 4075 & 3751\\ 
DMU17 & 30$\times$20 & 4142 & 4262 & 3814\\ 
DMU18 & 30$\times$20 & 4074 & 4174 & 3844\\ 
DMU19 & 30$\times$20 & 4033 & 4170 & 3764\\ 
DMU20 & 30$\times$20 & 3939 & 4038 & 3703\\ 
DMU21 & 40$\times$15 & 4408 & 4503 & 4380\\ 
DMU22 & 40$\times$15 & 4738 & 4819 & 4725\\ 
DMU23 & 40$\times$15 & \textbf{4668} & 4743 & 4668\\ 
DMU24 & 40$\times$15 & \textbf{4648} & 4676 & 4648\\ 
DMU25 & 40$\times$15 & \textbf{4164} & 4164 & 4164\\ 
DMU26 & 40$\times$20 & 5027 & 5131 & 4647\\ 
DMU27 & 40$\times$20 & 5038 & 5171 & 4848\\ 
DMU28 & 40$\times$20 & 4897 & 5035 & 4692\\ 
DMU29 & 40$\times$20 & 4924 & 5067 & 4691\\ 
DMU30 & 40$\times$20 & 4932 & 5093 & 4732\\ 
DMU31 & 50$\times$15 & \textbf{5640} & 5649 & 5640\\ 
DMU32 & 50$\times$15 & \textbf{5927} & 5927 & 5927\\ 
DMU33 & 50$\times$15 & \textbf{5728} & 5728 & 5728\\ 
DMU34 & 50$\times$15 & \textbf{5385} & 5385 & 5385\\ 
DMU35 & 50$\times$15 & \textbf{5635} & 5635 & 5635\\ 
DMU36 & 50$\times$20 & 5819 & 5981 & 5621\\ 
DMU37 & 50$\times$20 & 5991 & 6218 & 5851\\ 
DMU38 & 50$\times$20 & 6054 & 6199 & 5713\\ 
DMU39 & 50$\times$20 & 5773 & 5875 & 5747\\ 
DMU40 & 50$\times$20 & 5700 & 5815 & 5577\\ \bottomrule
\end{tabular}
\end{table}

\begin{table}[H]
\centering
\begin{tabular}{@{}ccccc@{}}
\toprule
\multirow{2}{*}{Instancia} & \multirow{2}{*}{Tamaño} & \multicolumn{2}{c}{ILS con N7} & \multirow{2}{*}{Estado del arte} \\ \cmidrule(lr){3-4}
& & Mejor& Mediana & \\ \midrule
DMU41 & 20$\times$15 & 3613 & 3791 & 3248\\ 
DMU42 & 20$\times$15 & 3750 & 3978 & 3390\\ 
DMU43 & 20$\times$15 & 3782 & 4058 & 3441\\ 
DMU44 & 20$\times$15 & 3891 & 4109 & 3475\\ 
DMU45 & 20$\times$15 & 3647 & 3962 & 3266\\ 
DMU46 & 20$\times$20 & 4462 & 4664 & 4035\\ 
DMU47 & 20$\times$20 & 4372 & 4545 & 3942\\ 
DMU48 & 20$\times$20 & 4200 & 4360 & 3763\\ 
DMU49 & 20$\times$20 & 4117 & 4311 & 3710\\ 
DMU50 & 20$\times$20 & 4352 & 4560 & 3729\\ 
DMU51 & 30$\times$15 & 4695 & 5063 & 4156\\ 
DMU52 & 30$\times$15 & 4895 & 5195 & 4303\\ 
DMU53 & 30$\times$15 & 5098 & 5399 & 4378\\ 
DMU54 & 30$\times$15 & 4933 & 5247 & 4361\\ 
DMU55 & 30$\times$15 & 4831 & 5142 & 4263\\ 
DMU56 & 30$\times$20 & 5910 & 6115 & 4941\\ 
DMU57 & 30$\times$20 & 5587 & 5842 & 4653\\ 
DMU58 & 30$\times$20 & 5619 & 5825 & 4701\\ 
DMU59 & 30$\times$20 & 5562 & 5881 & 4616\\ 
DMU60 & 30$\times$20 & 5759 & 5936 & 4721\\ 
DMU61 & 40$\times$15 & 5893 & 6366 & 5171\\ 
DMU62 & 40$\times$15 & 5943 & 6492 & 5248\\ 
DMU63 & 40$\times$15 & 6095 & 6447 & 5313\\ 
DMU64 & 40$\times$15 & 6044 & 6372 & 5226\\ 
DMU65 & 40$\times$15 & 5915 & 6181 & 5184\\ 
DMU66 & 40$\times$20 & 6870 & 7229 & 5701\\ 
DMU67 & 40$\times$20 & 7066 & 7379 & 5779\\ 
DMU68 & 40$\times$20 & 7175 & 7501 & 5763\\ 
DMU69 & 40$\times$20 & 6914 & 7309 & 5688\\ 
DMU70 & 40$\times$20 & 7148 & 7385 & 5868\\ 
DMU71 & 50$\times$15 & 7186 & 7449 & 6207\\ 
DMU72 & 50$\times$15 & 7381 & 7758 & 6463\\ 
DMU73 & 50$\times$15 & 7175 & 7437 & 6136\\ 
DMU74 & 50$\times$15 & 7175 & 7552 & 6196\\ 
DMU75 & 50$\times$15 & 7388 & 7644 & 6189\\ 
DMU76 & 50$\times$20 & 8620 & 8905 & 6718\\ 
DMU77 & 50$\times$20 & 8569 & 9032 & 6747\\ 
DMU78 & 50$\times$20 & 8796 & 9286 & 6755\\ 
DMU79 & 50$\times$20 & 8817 & 9149 & 6910\\ 
DMU80 & 50$\times$20 & 8287 & 8845 & 6634\\ \bottomrule
\end{tabular}
\end{table}
\label{app:resn7ils} \newpage
\section{Resultados para PN7Tup}

\begin{table}[H]
\centering
\begin{tabular}{@{}ccccc@{}}
\toprule
\multirow{2}{*}{Instancia} & \multirow{2}{*}{Tamaño} & \multicolumn{3}{c}{ PN7Tup} \\ \cmidrule(lr){3-5}
& & Mediana& Error relativo & Mejor  \\ \midrule
DMU01 & 20$\times$15 & 2834 & 0.11 & 2736\\ 
DMU02 & 20$\times$15 & 3006 & 0.11 & 2898\\ 
DMU03 & 20$\times$15 & 2982 & 0.09 & 2877\\ 
DMU04 & 20$\times$15 & 2913 & 0.09 & 2796\\ 
DMU05 & 20$\times$15 & 3066 & 0.12 & 2931\\ 
DMU06 & 20$\times$20 & 3547 & 0.09 & 3436\\ 
DMU07 & 20$\times$20 & 3360 & 0.10 & 3233\\ 
DMU08 & 20$\times$20 & 3427 & 0.07 & 3327\\ 
DMU09 & 20$\times$20 & 3353 & 0.08 & 3208\\ 
DMU10 & 20$\times$20 & 3196 & 0.07 & 3095\\ 
DMU11 & 30$\times$15 & 3804 & 0.11 & 3686\\ 
DMU12 & 30$\times$15 & 3938 & 0.13 & 3711\\ 
DMU13 & 30$\times$15 & 4088 & 0.11 & 3952\\ 
DMU14 & 30$\times$15 & 3634 & 0.07 & 3543\\ 
DMU15 & 30$\times$15 & 3545 & 0.06 & 3440\\ 
DMU16 & 30$\times$20 & 4044 & 0.08 & 3952\\ 
DMU17 & 30$\times$20 & 4249 & 0.11 & 4134\\ 
DMU18 & 30$\times$20 & 4164 & 0.08 & 4073\\ 
DMU19 & 30$\times$20 & 4184 & 0.11 & 4041\\ 
DMU20 & 30$\times$20 & 4042 & 0.09 & 3918\\ 
DMU21 & 40$\times$15 & 4504 & 0.03 & 4418\\ 
DMU22 & 40$\times$15 & 4769 & 0.01 & 4738\\ 
DMU23 & 40$\times$15 & 4743 & 0.02 & 4685\\ 
DMU24 & 40$\times$15 & 4669 & 0.00 & \textbf{4648}\\ 
DMU25 & 40$\times$15 & 4164 & 0.00 & \textbf{4164}\\ 
DMU26 & 40$\times$20 & 5111 & 0.10 & 4963\\ 
DMU27 & 40$\times$20 & 5176 & 0.07 & 5006\\ 
DMU28 & 40$\times$20 & 5014 & 0.07 & 4887\\ 
DMU29 & 40$\times$20 & 5086 & 0.08 & 4939\\ 
DMU30 & 40$\times$20 & 5053 & 0.07 & 4929\\ 
DMU31 & 50$\times$15 & 5646 & 0.00 & \textbf{5640}\\ 
DMU32 & 50$\times$15 & 5927 & 0.00 & \textbf{5927}\\ 
DMU33 & 50$\times$15 & 5728 & 0.00 & \textbf{5728}\\ 
DMU34 & 50$\times$15 & 5385 & 0.00 & \textbf{5385}\\ 
DMU35 & 50$\times$15 & 5635 & 0.00 & \textbf{5635}\\ 
DMU36 & 50$\times$20 & 5980 & 0.06 & 5880\\ 
DMU37 & 50$\times$20 & 6178 & 0.06 & 6045\\ 
DMU38 & 50$\times$20 & 6204 & 0.09 & 6008\\ 
DMU39 & 50$\times$20 & 5888 & 0.02 & 5786\\ 
DMU40 & 50$\times$20 & 5826 & 0.04 & 5711\\ \bottomrule
\end{tabular}
\end{table}

\begin{table}[H]
\centering
\begin{tabular}{@{}ccccc@{}}
\toprule
\multirow{2}{*}{Instancia} & \multirow{2}{*}{Tamaño} & \multicolumn{3}{c}{ PN7Tup} \\ \cmidrule(lr){3-5}
& & Mediana& Error relativo & Mejor  \\ \midrule
DMU41 & 20$\times$15 & 3795 & 0.17 & 3626\\ 
DMU42 & 20$\times$15 & 3985 & 0.18 & 3712\\ 
DMU43 & 20$\times$15 & 4016 & 0.17 & 3813\\ 
DMU44 & 20$\times$15 & 4075 & 0.17 & 3856\\ 
DMU45 & 20$\times$15 & 3935 & 0.20 & 3640\\ 
DMU46 & 20$\times$20 & 4601 & 0.14 & 4418\\ 
DMU47 & 20$\times$20 & 4514 & 0.15 & 4361\\ 
DMU48 & 20$\times$20 & 4346 & 0.15 & 4183\\ 
DMU49 & 20$\times$20 & 4331 & 0.17 & 4070\\ 
DMU50 & 20$\times$20 & 4509 & 0.21 & 4292\\ 
DMU51 & 30$\times$15 & 5045 & 0.21 & 4708\\ 
DMU52 & 30$\times$15 & 5213 & 0.21 & 4901\\ 
DMU53 & 30$\times$15 & 5381 & 0.23 & 5039\\ 
DMU54 & 30$\times$15 & 5312 & 0.22 & 4883\\ 
DMU55 & 30$\times$15 & 5125 & 0.20 & 4837\\ 
DMU56 & 30$\times$20 & 6044 & 0.22 & 5808\\ 
DMU57 & 30$\times$20 & 5817 & 0.25 & 5508\\ 
DMU58 & 30$\times$20 & 5772 & 0.23 & 5518\\ 
DMU59 & 30$\times$20 & 5854 & 0.27 & 5455\\ 
DMU60 & 30$\times$20 & 5906 & 0.25 & 5717\\ 
DMU61 & 40$\times$15 & 6269 & 0.21 & 6040\\ 
DMU62 & 40$\times$15 & 6462 & 0.23 & 6039\\ 
DMU63 & 40$\times$15 & 6465 & 0.22 & 6005\\ 
DMU64 & 40$\times$15 & 6424 & 0.23 & 6036\\ 
DMU65 & 40$\times$15 & 6187 & 0.19 & 5814\\ 
DMU66 & 40$\times$20 & 7072 & 0.24 & 6843\\ 
DMU67 & 40$\times$20 & 7313 & 0.27 & 6987\\ 
DMU68 & 40$\times$20 & 7438 & 0.29 & 6978\\ 
DMU69 & 40$\times$20 & 7240 & 0.27 & 6857\\ 
DMU70 & 40$\times$20 & 7339 & 0.25 & 7088\\ 
DMU71 & 50$\times$15 & 7457 & 0.20 & 7022\\ 
DMU72 & 50$\times$15 & 7729 & 0.20 & 7427\\ 
DMU73 & 50$\times$15 & 7443 & 0.21 & 7170\\ 
DMU74 & 50$\times$15 & 7511 & 0.21 & 7216\\ 
DMU75 & 50$\times$15 & 7637 & 0.23 & 7231\\ 
DMU76 & 50$\times$20 & 8832 & 0.31 & 8538\\ 
DMU77 & 50$\times$20 & 8938 & 0.32 & 8654\\ 
DMU78 & 50$\times$20 & 9058 & 0.34 & 8670\\ 
DMU79 & 50$\times$20 & 9068 & 0.31 & 8678\\ 
DMU80 & 50$\times$20 & 8711 & 0.31 & 8467\\ \bottomrule
\end{tabular}
\end{table}
\label{app:resn7tuple} \newpage
\section{Resultados para PN7extTup}

\begin{table}[H]
\centering
\begin{tabular}{@{}ccccc@{}}
\toprule
\multirow{2}{*}{Instancia} & \multirow{2}{*}{Tamaño} & \multicolumn{3}{c}{PN7extTup} \\ \cmidrule(lr){3-5}
& & Mediana& Error relativo & Mejor  \\ \midrule
DMU01 & 20$\times$15 & 2743 & 0.07 & 2640\\ 
DMU02 & 20$\times$15 & 2858 & 0.06 & 2782\\ 
DMU03 & 20$\times$15 & 2863 & 0.05 & 2788\\ 
DMU04 & 20$\times$15 & 2784 & 0.04 & 2687\\ 
DMU05 & 20$\times$15 & 2903 & 0.06 & 2822\\ 
DMU06 & 20$\times$20 & 3412 & 0.05 & 3320\\ 
DMU07 & 20$\times$20 & 3252 & 0.07 & 3183\\ 
DMU08 & 20$\times$20 & 3312 & 0.04 & 3256\\ 
DMU09 & 20$\times$20 & 3245 & 0.05 & 3154\\ 
DMU10 & 20$\times$20 & 3102 & 0.04 & 3061\\ 
DMU11 & 30$\times$15 & 3696 & 0.08 & 3608\\ 
DMU12 & 30$\times$15 & 3744 & 0.07 & 3651\\ 
DMU13 & 30$\times$15 & 3946 & 0.07 & 3805\\ 
DMU14 & 30$\times$15 & 3528 & 0.04 & 3435\\ 
DMU15 & 30$\times$15 & 3457 & 0.03 & 3386\\ 
DMU16 & 30$\times$20 & 3986 & 0.06 & 3924\\ 
DMU17 & 30$\times$20 & 4162 & 0.09 & 4093\\ 
DMU18 & 30$\times$20 & 4123 & 0.07 & 4028\\ 
DMU19 & 30$\times$20 & 4083 & 0.08 & 3970\\ 
DMU20 & 30$\times$20 & 3961 & 0.07 & 3865\\ 
DMU21 & 40$\times$15 & 4459 & 0.02 & 4390\\ 
DMU22 & 40$\times$15 & 4743 & 0.00 & \textbf{4725}\\ 
DMU23 & 40$\times$15 & 4736 & 0.01 & \textbf{4668}\\ 
DMU24 & 40$\times$15 & 4653 & 0.00 & \textbf{4648}\\ 
DMU25 & 40$\times$15 & 4164 & 0.00 & \textbf{4164}\\ 
DMU26 & 40$\times$20 & 5127 & 0.10 & 4987\\ 
DMU27 & 40$\times$20 & 5175 & 0.07 & 5017\\ 
DMU28 & 40$\times$20 & 5007 & 0.07 & 4924\\ 
DMU29 & 40$\times$20 & 5086 & 0.08 & 4962\\ 
DMU30 & 40$\times$20 & 5050 & 0.07 & 4951\\ 
DMU31 & 50$\times$15 & 5677 & 0.01 & \textbf{5640}\\ 
DMU32 & 50$\times$15 & 5927 & 0.00 & \textbf{5927}\\ 
DMU33 & 50$\times$15 & 5728 & 0.00 & \textbf{5728}\\ 
DMU34 & 50$\times$15 & 5385 & 0.00 & \textbf{5385}\\ 
DMU35 & 50$\times$15 & 5635 & 0.00 & \textbf{5635}\\ 
DMU36 & 50$\times$20 & 6036 & 0.07 & 5913\\ 
DMU37 & 50$\times$20 & 6270 & 0.07 & 6140\\ 
DMU38 & 50$\times$20 & 6317 & 0.11 & 6154\\ 
DMU39 & 50$\times$20 & 5947 & 0.03 & 5859\\ 
DMU40 & 50$\times$20 & 5952 & 0.07 & 5842\\ \bottomrule
\end{tabular}
\end{table}

\begin{table}[H]
\centering
\begin{tabular}{@{}ccccc@{}}
\toprule
\multirow{2}{*}{Instancia} & \multirow{2}{*}{Tamaño} & \multicolumn{3}{c}{PN7extTup} \\ \cmidrule(lr){3-5}
& & Mediana& Error relativo & Mejor  \\ \midrule
DMU41 & 20$\times$15 & 3601 & 0.11 & 3449\\ 
DMU42 & 20$\times$15 & 3725 & 0.10 & 3557\\ 
DMU43 & 20$\times$15 & 3810 & 0.11 & 3673\\ 
DMU44 & 20$\times$15 & 3806 & 0.10 & 3647\\ 
DMU45 & 20$\times$15 & 3621 & 0.11 & 3424\\ 
DMU46 & 20$\times$20 & 4455 & 0.10 & 4361\\ 
DMU47 & 20$\times$20 & 4369 & 0.11 & 4250\\ 
DMU48 & 20$\times$20 & 4147 & 0.10 & 4024\\ 
DMU49 & 20$\times$20 & 4106 & 0.11 & 3935\\ 
DMU50 & 20$\times$20 & 4347 & 0.17 & 4047\\ 
DMU51 & 30$\times$15 & 4893 & 0.18 & 4650\\ 
DMU52 & 30$\times$15 & 5023 & 0.17 & 4779\\ 
DMU53 & 30$\times$15 & 5206 & 0.19 & 4998\\ 
DMU54 & 30$\times$15 & 5114 & 0.17 & 4876\\ 
DMU55 & 30$\times$15 & 4887 & 0.15 & 4750\\ 
DMU56 & 30$\times$20 & 5964 & 0.21 & 5756\\ 
DMU57 & 30$\times$20 & 5678 & 0.22 & 5397\\ 
DMU58 & 30$\times$20 & 5670 & 0.21 & 5435\\ 
DMU59 & 30$\times$20 & 5600 & 0.21 & 5418\\ 
DMU60 & 30$\times$20 & 5773 & 0.22 & 5543\\ 
DMU61 & 40$\times$15 & 6410 & 0.24 & 6081\\ 
DMU62 & 40$\times$15 & 6499 & 0.24 & 6075\\ 
DMU63 & 40$\times$15 & 6520 & 0.23 & 6256\\ 
DMU64 & 40$\times$15 & 6479 & 0.24 & 6228\\ 
DMU65 & 40$\times$15 & 6290 & 0.21 & 6033\\ 
DMU66 & 40$\times$20 & 7228 & 0.27 & 6902\\ 
DMU67 & 40$\times$20 & 7370 & 0.28 & 7118\\ 
DMU68 & 40$\times$20 & 7447 & 0.29 & 7234\\ 
DMU69 & 40$\times$20 & 7275 & 0.28 & 6950\\ 
DMU70 & 40$\times$20 & 7408 & 0.26 & 7200\\ 
DMU71 & 50$\times$15 & 7870 & 0.27 & 7482\\ 
DMU72 & 50$\times$15 & 8168 & 0.26 & 7799\\ 
DMU73 & 50$\times$15 & 7923 & 0.29 & 7478\\ 
DMU74 & 50$\times$15 & 8007 & 0.29 & 7550\\ 
DMU75 & 50$\times$15 & 8000 & 0.29 & 7663\\ 
DMU76 & 50$\times$20 & 9071 & 0.35 & 8763\\ 
DMU77 & 50$\times$20 & 9046 & 0.34 & 8765\\ 
DMU78 & 50$\times$20 & 9146 & 0.35 & 8882\\ 
DMU79 & 50$\times$20 & 9268 & 0.34 & 8887\\ 
DMU80 & 50$\times$20 & 8883 & 0.34 & 8574\\ \bottomrule
\end{tabular}
\end{table}
\label{app:resn8tuple} \newpage
\section{Resultados para RpKeTup}

\begin{table}[H]
\centering
\begin{tabular}{@{}ccccc@{}}
\toprule
\multirow{2}{*}{Instancia} & \multirow{2}{*}{Tamaño} & \multicolumn{3}{c}{RpKeTup} \\ \cmidrule(lr){3-5}
& & Mediana& Error relativo & Mejor  \\ \midrule
DMU01 & 20$\times$15 & 2746 & 0.07 & 2651\\ 
DMU02 & 20$\times$15 & 2925 & 0.08 & 2824\\ 
DMU03 & 20$\times$15 & 2914 & 0.07 & 2828\\ 
DMU04 & 20$\times$15 & 2834 & 0.06 & 2736\\ 
DMU05 & 20$\times$15 & 2942 & 0.07 & 2848\\ 
DMU06 & 20$\times$20 & 3523 & 0.09 & 3366\\ 
DMU07 & 20$\times$20 & 3255 & 0.07 & 3171\\ 
DMU08 & 20$\times$20 & 3443 & 0.08 & 3319\\ 
DMU09 & 20$\times$20 & 3349 & 0.08 & 3225\\ 
DMU10 & 20$\times$20 & 3183 & 0.07 & 3109\\ 
DMU11 & 30$\times$15 & 3743 & 0.09 & 3598\\ 
DMU12 & 30$\times$15 & 3826 & 0.10 & 3715\\ 
DMU13 & 30$\times$15 & 3969 & 0.08 & 3793\\ 
DMU14 & 30$\times$15 & 3596 & 0.06 & 3459\\ 
DMU15 & 30$\times$15 & 3505 & 0.05 & 3410\\ 
DMU16 & 30$\times$20 & 4105 & 0.09 & 3982\\ 
DMU17 & 30$\times$20 & 4237 & 0.11 & 4077\\ 
DMU18 & 30$\times$20 & 4209 & 0.10 & 4035\\ 
DMU19 & 30$\times$20 & 4122 & 0.10 & 3946\\ 
DMU20 & 30$\times$20 & 4043 & 0.09 & 3894\\ 
DMU21 & 40$\times$15 & 4517 & 0.03 & 4405\\ 
DMU22 & 40$\times$15 & 4824 & 0.02 & \textbf{4725}\\ 
DMU23 & 40$\times$15 & 4738 & 0.01 & 4685\\ 
DMU24 & 40$\times$15 & 4687 & 0.01 & \textbf{4648}\\ 
DMU25 & 40$\times$15 & 4170 & 0.00 & \textbf{4164}\\ 
DMU26 & 40$\times$20 & 5089 & 0.10 & 4948\\ 
DMU27 & 40$\times$20 & 5236 & 0.08 & 5046\\ 
DMU28 & 40$\times$20 & 5072 & 0.08 & 4937\\ 
DMU29 & 40$\times$20 & 5188 & 0.11 & 4945\\ 
DMU30 & 40$\times$20 & 5089 & 0.08 & 4954\\ 
DMU31 & 50$\times$15 & 5821 & 0.03 & 5641\\ 
DMU32 & 50$\times$15 & 5927 & 0.00 & \textbf{5927}\\ 
DMU33 & 50$\times$15 & 5728 & 0.00 & \textbf{5728}\\ 
DMU34 & 50$\times$15 & 5443 & 0.01 & \textbf{5385}\\ 
DMU35 & 50$\times$15 & 5645 & 0.00 & \textbf{5635}\\ 
DMU36 & 50$\times$20 & 6164 & 0.10 & 6023\\ 
DMU37 & 50$\times$20 & 6311 & 0.08 & 6169\\ 
DMU38 & 50$\times$20 & 6248 & 0.09 & 5985\\ 
DMU39 & 50$\times$20 & 6099 & 0.06 & 5946\\ 
DMU40 & 50$\times$20 & 6080 & 0.09 & 5844\\ \bottomrule
\end{tabular}
\end{table}

\begin{table}[H]
\centering
\begin{tabular}{@{}ccccc@{}}
\toprule
\multirow{2}{*}{Instancia} & \multirow{2}{*}{Tamaño} & \multicolumn{3}{c}{RpKeTup} \\ \cmidrule(lr){3-5}
& & Mediana& Error relativo & Mejor  \\ \midrule
DMU41 & 20$\times$15 & 3560 & 0.10 & 3461\\ 
DMU42 & 20$\times$15 & 3743 & 0.10 & 3580\\ 
DMU43 & 20$\times$15 & 3782 & 0.10 & 3626\\ 
DMU44 & 20$\times$15 & 3862 & 0.11 & 3692\\ 
DMU45 & 20$\times$15 & 3596 & 0.10 & 3467\\ 
DMU46 & 20$\times$20 & 4479 & 0.11 & 4303\\ 
DMU47 & 20$\times$20 & 4314 & 0.09 & 4178\\ 
DMU48 & 20$\times$20 & 4140 & 0.10 & 4006\\ 
DMU49 & 20$\times$20 & 4102 & 0.11 & 3902\\ 
DMU50 & 20$\times$20 & 4166 & 0.12 & 4026\\ 
DMU51 & 30$\times$15 & 4741 & 0.14 & 4553\\ 
DMU52 & 30$\times$15 & 4843 & 0.13 & 4726\\ 
DMU53 & 30$\times$15 & 4966 & 0.13 & 4762\\ 
DMU54 & 30$\times$15 & 4975 & 0.14 & 4781\\ 
DMU55 & 30$\times$15 & 4784 & 0.12 & 4659\\ 
DMU56 & 30$\times$20 & 5646 & 0.14 & 5448\\ 
DMU57 & 30$\times$20 & 5312 & 0.14 & 5153\\ 
DMU58 & 30$\times$20 & 5348 & 0.14 & 5159\\ 
DMU59 & 30$\times$20 & 5300 & 0.15 & 5103\\ 
DMU60 & 30$\times$20 & 5399 & 0.14 & 5222\\ 
DMU61 & 40$\times$15 & 6034 & 0.17 & 5850\\ 
DMU62 & 40$\times$15 & 5995 & 0.14 & 5779\\ 
DMU63 & 40$\times$15 & 6096 & 0.15 & 5901\\ 
DMU64 & 40$\times$15 & 6065 & 0.16 & 5863\\ 
DMU65 & 40$\times$15 & 5953 & 0.15 & 5717\\ 
DMU66 & 40$\times$20 & 6677 & 0.17 & 6491\\ 
DMU67 & 40$\times$20 & 6832 & 0.18 & 6570\\ 
DMU68 & 40$\times$20 & 6729 & 0.17 & 6532\\ 
DMU69 & 40$\times$20 & 6695 & 0.18 & 6457\\ 
DMU70 & 40$\times$20 & 6983 & 0.19 & 6736\\ 
DMU71 & 50$\times$15 & 7387 & 0.19 & 7155\\ 
DMU72 & 50$\times$15 & 7563 & 0.17 & 7295\\ 
DMU73 & 50$\times$15 & 7257 & 0.18 & 6999\\ 
DMU74 & 50$\times$15 & 7320 & 0.18 & 7060\\ 
DMU75 & 50$\times$15 & 7347 & 0.19 & 7089\\ 
DMU76 & 50$\times$20 & 8111 & 0.21 & 7819\\ 
DMU77 & 50$\times$20 & 8138 & 0.21 & 7852\\ 
DMU78 & 50$\times$20 & 8219 & 0.22 & 7926\\ 
DMU79 & 50$\times$20 & 8344 & 0.21 & 7915\\ 
DMU80 & 50$\times$20 & 7975 & 0.20 & 7753\\ \bottomrule
\end{tabular}
\end{table}
\label{app:resprtuple} \newpage

\bibliographystyle{acm}
\bibliography{Referencias}

\end{document}
