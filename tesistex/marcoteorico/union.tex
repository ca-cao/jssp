%Como se juntan las definiciones con lo demás
\section{Metaheurísticas aplicadas al JSP}
La complejidad del JSP hace que las metaheurísticas sean actualmente los métodos más utilizados para resolver instancias grandes. Estas han conseguido hallar buenas soluciones para los conjuntos de prueba más populares. Se han empleado una gran variedad de metaheurísticas como: algoritmos genéticos\cite{Cheng1999}, evolución diferencial\cite{Ponsich2013}, algoritmos de colonia de hormigas\cite{Klc2007}, algoritmos de enjambre de partículas\cite{zhang2019novel} y búsqueda tabú \cite{Zhang2007}. Aunque estos son algoritmos muy diferentes entre sí, la mayoría de ellos aplican de alguna forma el concepto de búsqueda local para mejorar su desempeño.\\

Para aplicar el proceso de búsqueda local es necesario introducir una definición de vecindad, como se ha mencionado anteriormente, esta estructura es parte del paisaje de búsqueda y tiene un gran impacto en los resultados obtenidos. Desde que se propuso la primera vecindad en 1996, se han propuesto vecindades cada vez con mejores resultados. A continuación se presentan las vecindades más importantes hasta la fecha. 


\subsection*{Vecindades previamente propuestas}
Todas las vecindades presentadas aquí se basan en el concepto de ruta crítica y solo consideran cambios dentro de los bloques críticos. En las figuras siguientes se presentan figuras ilustrativas para cada una de estas definiciones.

\begin{itemize}
\item N1 \cite{blazewicz1996job} Consiste en considerar todas las soluciones que se crean al intercambiar cualquier par de operaciones adyacentes que pertenecen a un bloque crítico. Aunque esta vecindad no es particularmente grande ya que es aproximadamente del tamaño de la ruta crítca, considera muchos cambios que no mejoran el makespan.
\begin{figure}[H]
\centering
\includegraphics[scale=.7]{Imagenes/N1.pdf}
\caption{Movimientos de la vecindad N1}
\end{figure}

\item N4 \cite{dell1993applying} Esta vecindad se propuso como un refinamiento y extensión de la vecindad N1 y toma como base el concepto de bloque crítico. Consiste en llevar operaciones internas del bloque crítico al inicio o final. Esta vecindad es de tamaño comparable a la anterior pero presenta movimientos más perturbativos.
\begin{figure}[H]
\centering
\includegraphics[scale=.7]{Imagenes/N4.pdf}
\caption{Movimientos de la vecindad N4}
\end{figure}


\item N5 \cite{EugeniuszNowicki2003} Consiste en intercambiar solo las operaciones adyacentes a la final o inicial de un bloque crítico.  Es aun más pequeña que la anterior ya que es aproximadamente del tamaño del número de bloques críticos.  
\begin{figure}[H]
\centering
\includegraphics[scale=.7]{Imagenes/N5.pdf}
\caption{Movimientos de la vecindad N5}
\end{figure}

\item N6 \cite{Balas1998} Los autores utilizan varios teoremas para identificar pares $(u,v)$ de operaciones dentro de un bloque crítico que puedan llevar a mejorar la solución y a su vez identificar si se tiene que mover a $u$ justo después de $v$(forward) o bien a $v$ justo antes de $u$ (backward). Esta vecindad puede verse com un refinamiento y extensión de la N4 por lo que tiene un tamaño similar.
\begin{figure}[H]
\centering
\includegraphics[scale=.7]{Imagenes/N6.pdf}
\caption{Los dos tipos de movimientos para un par $(u,v)$}
\end{figure}

\item N7 \cite{Zhang2007} Esta vecindad se plantea como una extensión de la N6 en la cual se toma la idea de los movimientos entre pares de operaciones de un bloque crítico. Los autores toman en cuenta todos los cambios posibles entre el inicio o fin del bloque crítico con todas las operaciones internas.

\begin{figure}[H]
\centering
\includegraphics[scale=.7]{Imagenes/N7.pdf}
\caption{Movimientos de la vecindad N7}
\end{figure}
\end{itemize}

Las vecindades antes presentadas se han centrado en operaciones que pertenecen a la ruta crítica, esto les confiere las siguientes ventajas: la vecindad resultante es suficientemente pequeña como para ser explorada en su totalidad, para deducir el makespan de una planificación necesariamente deben hacerse cambios en la ruta crítica y puede garantizarse que cualquier vecino representa una solución factible\cite{balas1969machine}.\\

La literatura reciente se ha centrado en hacer los algoritmos existentes más eficientes dejando de lado la parte de la representación o de la función de fitness. Existen otras ideas que no se han explorado a fondo pero parecen prometedoras como proponer extensiones a alguna de las vecindades, cambiar la representación y proponer una función de fitness que no solo tome en cuenta el makespan de modo que tengamos más formas de diferenciar entre soluciones. En la siguiente sección se presentan algunas otras medidas de calidad para planificaciones del JSP.
% ve como se pueden mejorar 
\subsection*{Criterios de optimalidad}
Como ya se ha mencionado el criterio de optimalidad más usado es el makespan, no obstante existen muchos otros criterios de optimalidad que pueden usarse para asignar un valor de fitness a una planificación. 

Si denotamos por $C_i$ al tiempo de finalización de la máquina $i$ y por $f_i(C_i)$ a su costo asociado, podemos distinguir dos tipos de funciones de costo en la literatura\cite{Brucker2001}:
\[f_{\max}:=\max\{f_i(C_i)\}\]
y 
\[\sum f_i(C):=\sum_{1\leq i\leq n}f_i(C_i)\]

Los costos asociados a cada uno de los tiempos de finalización de los trabajos pueden tomar muchas formas, por ejemplo pueden introducirse pesos para cada trabajo o fijar tiempos de finalización esperado para cada trabajo y medir la desviación de ellos.

Dependiendo del problema en sí, puede ser que se le de más o menos valor a distintos aspectos de la planificación como el tiempo que están detenidas las máquinas, o el tiempo que tarda un trabajo en particular. El makespan se ha extendido como criterio de optimalidad porque está muy relacionado con los costos económicos de la planificación\cite{Rand1977}.


Con las definiciones anteriores sobre vecindades, función objetivo y representaciones de las soluciones podemos construir el paisaje de búsqueda para el JSP. 

\section{Paisaje de búsqueda del JSP}

Analizar el paisaje de búsqueda de un problema de optimización combinatoria puede ayudar a escoger o diseñar metaheurísticas aunque en la práctica resulta muy díficil de hacer por el tamaño del espacio de soluciones. Se ha observado que los problemas de optimización combinatoria a menudo presentan un un patrón \textit{fácil-difícil-fácil} \cite{mammen1997new} de acuerdo con la dificultad para cumplir las restricciones impuestas: el problema es fácil de resolver si tiene constricciones muy débiles o muy fuertes y es difícil si está un punto intermedio. La explicación más extendida para este fenómeno consiste en la competencia de dos factores: por un lado un aumento de las restricciones que implica una reducción de las soluciones factibles a la vez que causa una disminución en el número de conexiones entre soluciones.\\


Para el JSP se han realizado estudios con instancias pequeñas en los que se puede construir explícitamente todo el paisaje, i. e. se puede evaluar y construir la vecindad correspondiente a todas las soluciones. Con este paisaje construido pueden medirse cantidades de interés como la similitud entre óptimos locales o globales, el número de vecinos entre otros que dan pistas de la forma (como las mostradas en la figura \ref{fig:landtypes}) que tiene el paisaje de búsqueda.\\

Se ha encontrado que el paisaje de búsqueda para una instancia al azar sigue el patrón \textit{fácil-difícil-fácil} dependiendo de la razón entre el número de máquinas y número de trabajos de la misma\cite{Streeter2006}  ($\frac{N}{M}$) conforme esta cantidad  varía de 0 a infinito, siendo el caso más \textit{difícil} cuando $\frac{N}{M}\simeq 1$. También se ha encontrado este patrón al medir la fuerza de las constricciones la fuerza de las constricciones se mide al comparar el número de soluciones factibles contra el número de bits necesarios para representar una solución\cite{beck1997constrainedness}.\\

Los experimentos computacionales sugieren que este patrón se da por cambios en la rugosidad del paisaje de búsqueda que se vuelve más y más pronunciada conforme $\frac{N}{M}$ o la fuerza de las constricciones crecen, para valores pequeños se tiene un paisaje semejante a un gran valle suave, para valores grandes se tiene un pasaje con muchos óptimos locales de buena calidad y para valores cercanos a 1 se tienen muchos óptimos locales de baja calidad.\\


También se ha encontrado que una de las características que hacen difícil a una instancia del JSP es la existencia de muchos óptimos locales de baja calidad que no son parecidos entre sí \cite{mattfeld1999search}. Esto implica que sin importar el punto inicial de la metaheurística elegida rara vez estaremos muy lejos de uno de estos puntos. Otro factor que afecta el proceso de búsqueda local es que en general se ha encontrado que algunas soluciones están mucho mas conectadas que otras \cite{bierwirth2004landscape}. Este último hecho puede ser tanto benéfico como perjudicial dependiendo si las soluciones altamente conectadas son de buena o mala calidad. \\


Estas características explican en parte por qué los métodos basados en búsqueda local como la búsqueda tabú \ref{alg:TS} combinados con métodos sofisticados de exploración han tenido tan buenos resultados\cite{watson2003problem}. Puede ser que las soluciones que estén más conectadas a otras sean de baja calidad y sea necesario evitar ser <<atraídos>> hacia ellas para llegar a soluciones de mejor calidad. También se ha observado que el paisaje de búsqueda para el JSP presenta grandes <<planices>> i. e. zonas en las que todas las soluciones vecinas tienen el mismo makespan, en la práctica estas zonas actuan como un solo óptimo local que está altamente conectado.

En este sentido, podemos pensar que para que una metaheurística basada en búsqueda local tenga mayor probabilidad de encontrar buenas soluciones debemos plantear el paisaje de búsqueda de modo que no nos atasquemos en óptimos locales de mala calidad, ya sea porque están muy conectados y funcionan como atractores o bien porque el paisaje de búsqueda es tan rugoso que hay óptimos locales por doquier y es fácil atascarse en uno de mala calidad. Las soluciones altamente conectadas también pueden servir como un punto de partida hacia una solución de mejor calidad. 

% strogatz small world networks
% figuras de grafos con buenas y malas características
