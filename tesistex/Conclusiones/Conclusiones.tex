El problema de planificación tipo taller (JSP) es un problema de optimización combinatoria que ha recibido mucha atención desde hace varias décadas por sus aplicaciones 
en ambientes de manufactura. 
%
Este problema pertenece a la clase \textbf{NP-hard}, por lo que no se conocen algoritmos eficientes para resolverlo de manera general.
%
Es por ello que durante los últimos años, se han desarrollado múltiples optimizadores para el JSP basados en métodos aproximados. 
%
Dichos optimizadores han incrementando progresivamente su complejidad en implementación y mantenimiento, así como los requerimientos
de cómputo.

En la actualidad, los optimizadores que ofrecen los mejores resultados son técnicas híbridas que combinan múltiples ideas de muy diferente índole 
y requieren de grandes cantidades de recursos computacionales para su buen desempeño. 
%
En general el rendimiento de los optimizadores dependen del paisaje de búsqueda, es decir, de la representación que se le da a las soluciones, de la estructura de vecindad 
y de la función de fitness. 
%
Aunque se han usado diferentes tipos de representaciones, vecindades y funciones de fitness, gran parte de los esquemas actuales consideran aspectos comunes, y en particular, 
suelen usar la vecindad conocida como N7, una representación basada en permutaciones y el makespan como función de fitness. 
%
En cuanto a la forma de explorar el paisaje de búsqueda, actualmente los métodos más exitosos se basan en la metaheurística conocida como búsqueda tabú, pero incorporando 
múltiples modificaciones.

En esta tesis se propone un conjunto de modificaciones a cada una de las componentes del paisaje de búsqueda con el fin de analizar qué efecto tienen 
los mismos sobre el rendimiento de una metaheurística de trayectoria simple. 
%
En concreto, con el fin de obtener soluciones de alta calidad en tiempos reducidos, se usa una metaheurística simple: la búsqueda local iterada.
%
Entre las propuestas cabe destacar una manera de extender la vecindad N7 con movimientos que buscan aprovechar tiempos inactivos en la planificación,
nuevas funciones de fitness que permiten distinguir entre soluciones con el mismo makespan, una nueva representación basada en
llaves aleatorias a cada operación junto con un algoritmo para construir una solución a partir de ellas, que genera soluciones activas, y
una nueva estructura de vecindad asociada a esta última representación. 

Los resultados muestran que, para las instancias más difíciles de la actualidad, la utilización de un paisaje de búsqueda adecuado permite reducir
significativamente el error relativo, y en particular, se consiguió un optimizador que obtiene en menos de 5 minutos errores relativos no mayores a $0.21$.
%
Entre las modificaciones propuestas, la representación junto con la nueva estructura de vecindad, y las funciones de fitness fueron las que tuvieron mayor
efecto en los resultados.
%
La nueva representación es particularmente útil para la segunda mitad de las instancias DMU, las cuales son consideradas actualmente como las más retadoras.
%
Por otro lado, la función de fitness también es bastante importante.
%
En particular, debido a la existencia de múltiples zonas planas cuando solo se usa el makespan como función de fitness, el optimizador se queda bloqueado 
en óptimos de baja calidad. 
%
Sin embargo, al añadir características relevantes como lo son los tiempos de finalización de otras máquinas, surgen nuevas direcciones de mejora que permiten 
dirigir la búsqueda hacia óptimos locales de mejor calidad. 
%
Se probaron varias formas de extender la función de fitness y en particular, la que obtuvo mejor rendimiento consiste en construir una tupla con los tiempos 
ordenados de finalización de cada una de las máquinas y compararlas lexicográficamente para distinguir si una solución es mejor que otra. 
%
Esta función de fitness logró recudir la mediana del error relativo en todas las instancias y en varias de ellas se redujo aproximadamente a la mitad.
%
En lo que se refiere a la extensión de la vecindad N7 propuesta, no hay ventajas tan claras, siendo mejor en algunas instancias y significativamente peor en otras, por 
lo que se debe ver como una alternativa, en lugar de como una mejora.

Además de presentar los resultados, se realizaron algunos estudios para analizar los efectos que tienen los cambios introducidos en el paisaje de búsqueda.
%
En primer lugar, la nueva estructura de vecindad asociada a la representación de llaves, tiene una conectividad más homogénea, mientras que sin usar la misma,
las soluciones de menor makespan tienden a estar menos conectadas.
%
En segundo lugar, y probablemente más importante, la nueva vecindad muestra una mayor suavidad en el paisaje de búsqueda lo cual implica que las soluciones tienden 
a agruparse con soluciones de calidad similar, lo que se sabe que es muy bueno para el rendimiento de las metaheurísticas.
%

Nótese que en esta tesis, el paisaje de búsqueda sólo se integró con la metaheurística búsqueda local iterada, lo que permitió generar buenas soluciones en 
tiempos reducidos de solo 5 minutos.
%
Dadas las importantes mejoras obtenidas, se abre la puerta a incluir estos cambios con otras metaheurísticas de mayor complejidad y con metaheurísticas híbridas. 
%
Esto es un reto importante porque en su mayor parte, estas metaheurísticas incluyes pasos dependientes de la representación como cálculo aproximado de fitness, 
prohibiciones de ciertos tipos de movimientos, etc. por lo que no es para nada trivial realizar la integración, aunque parece muy prometedor.
%
Sería muy interesante plantear formas de aprovechar todo el desarrollo que se tiene para la representación basada en permutaciones y traducirlo a la representación 
propuesta en este trabajo o hacer que ambas representaciones trabajen en conjunto. 
%
Por último, la construcción de la función de fitness parece ser una gran área de oportunidad, y se podría incluir en metaheurísticas de mayor complejidad. 
