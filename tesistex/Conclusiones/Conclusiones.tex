\section{Conclusiones}
Los resultados obtenidos muestran una clara diferencia en cuestión de dificultad entre las instancias \textbf{DMU01-40} y las \textbf{DMU41-80}. La búsqueda local iterada con la vecindad N7 y la función de fitness propuesta muestra un desempeño muy diferente en estas dos mitades. En la primera mitad se obtienen resultados con bajo error relativo e inclusive en varias ocasiones se alcanzan los resultados del estado del arte mientras que en la segunda mitad el error relativo es bastante más alto y crece con el tamaño de la instancia considerada. 

La propuesta de extensión para la vecindad N7 no mostró ventajas importantes frente a la vecindad original y muestra el mismo comportamiento a pesar de plantearse como una posible manera de escapar de óptimos locales y de mantener soluciones más prometedoras. 

La función de fitness construida con las características presentadas muestra que es benéfico considerar otros factores de la planificación a demás del makespan para obtener mejores resultados. Aunque sí se tienen mejoras con la adición de estas característica es posible que exista una mejor manera de construir la función de fitness por ejemplo con una suma ponderada aunque eso implica introducir y determinar los pesos a usar para cada una de ellas lo cual puede complicar de más el algoritmo.

El cambio de representación tuvo en gran efecto positivo en el desempeño de la búsqueda local. Este cambio logra una disminución sustancial del error relativo en la segunda mitad de las instancias. Puede observarse en la figura \ref{fig:mattgraph} que en esta nueva estructura de vecindad el tamaño de la misma está mucho más relacionado con el makespan de la solución esto sugiere que las mejores soluciones están poco conectadas haciéndolas más difíciles de obtener. Un aspecto positivo de la nueva estructura de vecindad es que a pesar de presentar esta clara correlación entre makespan y conectividad, el tamaño de la misma no fluctúa tanto como con la vecindad N7 a demás de que es más grande por lo que los problemas derivados de esta correlación no son muy apreciables. Por último, la nueva vecindad muestra una mayor <<suavidad>> en los óptimos locales como puede verse en las figuras \ref{fig:bxp1} y \ref{fig:bxp2}. Esto también implica que las soluciones tienden más a agruparse con soluciones de calidad similar, esta característica del paisaje de búsqueda puede aprovecharse en el futuro.

La modificación que tuvo el mayor impacto positivo fue la introducción de la nueva representación junto con la nueva estructura de vecindad. Las propuestas aquí presentadas indican posibles modificaciones a los métodos actuales o un punto de partida para proponer algunos nuevos.
